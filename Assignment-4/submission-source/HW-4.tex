\documentclass{article}
\usepackage{amsmath}				% want AMS fonts
\usepackage{amssymb}                            % use AMS symbols
\usepackage{graphicx}
\usepackage{enumerate,url}
 %\usepackage[ruled]{algorithm2e}
%% preamble.tex
%% this should be included with a command like
%% %% preamble.tex
%% this should be included with a command like
%% %% preamble.tex
%% this should be included with a command like
%% \input{p}

\usepackage{amsfonts}
\usepackage{amsthm}
\usepackage{latexsym}
\usepackage{amsmath}
\usepackage{color}
\usepackage{comment}
\usepackage{enumitem}
\newcommand{\sk}{s}
\newtheorem{theorem}{Theorem}
\newtheorem{lesson}{Lesson}
\newtheorem{proposition}{Proposition}
\newtheorem{lemma}{Lemma}
\newtheorem{corollary}{Corollary}
\newtheorem{fact}{Fact}
\newtheorem*{claim}{Claim}
%\theoremstyle{definition}
\newtheorem{definition}{Definition}
\newtheorem{assumption}{Assumption}
\theoremstyle{remark}
\newtheorem{example}{Example}
\newtheorem*{remark}{Remark}

\DeclareSymbolFont{AMSb}{U}{msb}{m}{n}
\DeclareMathSymbol{\F}{\mathalpha}{AMSb}{"46}
\DeclareMathSymbol{\N}{\mathalpha}{AMSb}{"4E}
\DeclareMathSymbol{\R}{\mathalpha}{AMSb}{"52}
\DeclareMathSymbol{\X}{\mathalpha}{AMSb}{"58}
\DeclareMathSymbol{\Zz}{\mathalpha}{AMSb}{"5A}
\newcommand{\Z}[1]{{\ensuremath{\Zz_{#1}} }}
\newcommand{\Zs}[1]{\ensuremath{\Zz^{\ast}_{#1}}}
\newcommand{\Zn}{\Z{n}}
\newcommand{\Zns}{\Zs{n}}
\newcommand{\Zstar}[1]{\Zs{#1}}
\newcommand{\Zp}{{\Z{p}}}
\newcommand{\Zps}{\Zs{p}}
\newcommand{\Zqs}{\Zs{q}}
\newcommand{\Zq}{{\Z{q}}}
%\newcommand{\QR}{\mathop{\mathrm{QR}}\nolimits}
\newcommand{\ord}[1]{\mathop{\mathrm{ord}}({#1})}
\newcommand{\QR}[1]{\ensuremath{\textit{QR}_{#1}}}
\newcommand{\becomes}{:=}
\newcommand{\rem}[1]{\ensuremath{\ \operatorname{rem} #1}}  
\newcommand{\U}{{\mathcal{U}}}
\newcommand{\floor}[1]{\ensuremath{\lfloor{#1}\rfloor}}
\newcommand{\de}[1]{\ensuremath{\Delta{#1}}}
\newcommand{\js}[2]{\left( \frac{#1}{#2} \right)}

\renewcommand{\QR}{{\mbox{QR}}}
\newcommand{\QNR}{{\mbox{QNR}}}
\newcommand{\crt}{{\mbox{CRT}}}
\newcommand{\rsa}{{\mbox{RSA}}}
\newcommand{\rsamod}{{\mbox{RSA-modulus}}}


\newcommand{\greq}[1]{\stackrel{#1}{=}}
\newcommand{\hash}{\ensuremath{\mathcal{H}}}
\newcommand{\negl}{{\tt neg}}
\newcommand{\cindist}{\stackrel{c}{\approx}}
\newcommand{\A}{{\mathcal{A}}}
\newcommand{\B}{{\mathcal{B}}}

\newcommand{\gen}{\mathsf{Gen}}
\newcommand{\keygen}{\mathsf{KeyGen}}
\newcommand{\RSA}{\mathsf{RSA}}
\newcommand{\pk}{\mathsf{pk}}
\newcommand{\PK}{\mathsf{PK}}
\newcommand{\SK}{\mathsf{SK}}
\newcommand{\CT}{\mathsf{CT}}
\newcommand{\sign}{\mathsf{Sign}}
\newcommand{\verify}{\mathsf{Verify}}
\newcommand{\enc}{\mathsf{Enc}}
\newcommand{\dec}{\mathsf{Dec}}
\newcommand{\mac}{\mathsf{Tag}}

\newcommand{\ASETUP}{\textsf{ABE.Setup}}
\newcommand{\AEXTRACT}{\textsf{ABE.Extract}}
\newcommand{\AENCRYPT}{\textsf{ABE.Enc}}
\newcommand{\ADECRYPT}{\textsf{ABE.Dec}}
\newcommand{\construction}{\mathit{Construction}}
\newcommand{\domain}{\mathit{Domain}}

\def\matA{\mathbf{A}}
\def\matT{\mathbf{T}}
\def\matB{\mathbf{B}}
\def\matF{\mathbf{F}}
\def\matH{\mathbf{H}}
\def\matM{\mathbf{M}}
\def\matS{\mathbf{S}}

\def\veca{\mathbf{a}}
\def\vecb{\mathbf{b}}
\def\vecc{\mathbf{c}}
\def\vecd{\mathbf{d}}
\def\vece{\mathbf{e}}
\def\vecu{\mathbf{u}}
\def\vecr{\mathbf{r}}
\def\vecm{{m}}
\def\vecs{\mathbf{s}}
\def\vect{{t}}
\def\vecv{{v}}
\def\vecw{\mathbf{w}}
\def\vecx{\mathbf{x}}
\def\vecy{\mathbf{y}}
\def\vecgamma{\mathbf{\gamma}}
\def\halfq{\left\lfloor\frac{q}{2}\right\rfloor}

\newcommand{\ZQ}{\mathbb{Z}_q}
\newcommand{\G}{\mathbb{G}}
\def\Q{\mathbb{Q}}
\newcommand{\nat}{\mathbb{N}}
\newcommand{\intg}{\mathbb{Z}}
\newcommand{\detm}{\mathsf{det}}

\newcommand{\getsr}{\overset{R}{\leftarrow}}

\setlength{\oddsidemargin}{.25in}
\setlength{\evensidemargin}{.25in}
\setlength{\textwidth}{6in}
\setlength{\topmargin}{-0.4in}
\setlength{\textheight}{8.5in}

\newcommand{\handout}[5]{
  % \renewcommand{\thepage}{#1-\arabic{page}}
   \noindent
   \begin{center}
   \framebox{
      \vbox{
    \hbox to 5.78in { {\bf CS60065: Cryptography and Network Security} \hfill {\it #2 }}
       \vspace{4mm}
       \hbox to 5.78in { {\Large \hfill #5  \hfill} }
       \vspace{2mm}
       \hbox to 5.78in { {\it #3 \hfill #4} }
      }
   }
   \end{center}
   \vspace*{4mm}
}

\newcommand{\ho}[4]{\handout{#1}{#2}{{\sf Instructor:}
#3}{}{Handout #1: #4}}

\newcommand{\lnotes}[3]{\handout{#1}{#2}{{\sf Instructor:}
#3}{}{Lecture #1}}


\newcommand{\quiz}[3]{\handout{#1}{#2}{{\sf Instructor:} #3}{Total: 40 pts}{Mid-Term #1}}

\newcommand{\major}[3]{\handout{#1}{#2}{{\sf Instructor:}
#3}{}{Major #1}}

\newcommand{\solution}[1]{#1}
%\homework{number}{out}{due}{instructor}
\newcommand{\homework}[4]{\handout{HW #1}{#2}{{\sf Instructor:} #4}{{\sf Due:} #3}{\textbf{Assignment #1}}}
\newcommand{\homeworksol}[4]{\handout{HW #1}{#2}{Name: #4}{{\sf Due:} #3}{Assignment #1}}

%================================================
% problemset macros
%================================================
% count problems
\newcounter{solutioncount}
\setcounter{solutioncount}{0}
\newcommand{\problem}[1]{%
\addtocounter{solutioncount}{1}%
\section*{Problem \arabic{solutioncount}: #1}}

% lets you make alphabetical lists (at first-level of enumeration)
\newenvironment
  {alphabetize}{\renewcommand{\theenumi}{\alph{enumi}}\begin{enumerate}}
  {\end{enumerate}\renewcommand{\theenumi}{\arabic{enumi}}}


\usepackage{amsfonts}
\usepackage{amsthm}
\usepackage{latexsym}
\usepackage{amsmath}
\usepackage{color}
\usepackage{comment}
\usepackage{enumitem}
\newcommand{\sk}{s}
\newtheorem{theorem}{Theorem}
\newtheorem{lesson}{Lesson}
\newtheorem{proposition}{Proposition}
\newtheorem{lemma}{Lemma}
\newtheorem{corollary}{Corollary}
\newtheorem{fact}{Fact}
\newtheorem*{claim}{Claim}
%\theoremstyle{definition}
\newtheorem{definition}{Definition}
\newtheorem{assumption}{Assumption}
\theoremstyle{remark}
\newtheorem{example}{Example}
\newtheorem*{remark}{Remark}

\DeclareSymbolFont{AMSb}{U}{msb}{m}{n}
\DeclareMathSymbol{\F}{\mathalpha}{AMSb}{"46}
\DeclareMathSymbol{\N}{\mathalpha}{AMSb}{"4E}
\DeclareMathSymbol{\R}{\mathalpha}{AMSb}{"52}
\DeclareMathSymbol{\X}{\mathalpha}{AMSb}{"58}
\DeclareMathSymbol{\Zz}{\mathalpha}{AMSb}{"5A}
\newcommand{\Z}[1]{{\ensuremath{\Zz_{#1}} }}
\newcommand{\Zs}[1]{\ensuremath{\Zz^{\ast}_{#1}}}
\newcommand{\Zn}{\Z{n}}
\newcommand{\Zns}{\Zs{n}}
\newcommand{\Zstar}[1]{\Zs{#1}}
\newcommand{\Zp}{{\Z{p}}}
\newcommand{\Zps}{\Zs{p}}
\newcommand{\Zqs}{\Zs{q}}
\newcommand{\Zq}{{\Z{q}}}
%\newcommand{\QR}{\mathop{\mathrm{QR}}\nolimits}
\newcommand{\ord}[1]{\mathop{\mathrm{ord}}({#1})}
\newcommand{\QR}[1]{\ensuremath{\textit{QR}_{#1}}}
\newcommand{\becomes}{:=}
\newcommand{\rem}[1]{\ensuremath{\ \operatorname{rem} #1}}  
\newcommand{\U}{{\mathcal{U}}}
\newcommand{\floor}[1]{\ensuremath{\lfloor{#1}\rfloor}}
\newcommand{\de}[1]{\ensuremath{\Delta{#1}}}
\newcommand{\js}[2]{\left( \frac{#1}{#2} \right)}

\renewcommand{\QR}{{\mbox{QR}}}
\newcommand{\QNR}{{\mbox{QNR}}}
\newcommand{\crt}{{\mbox{CRT}}}
\newcommand{\rsa}{{\mbox{RSA}}}
\newcommand{\rsamod}{{\mbox{RSA-modulus}}}


\newcommand{\greq}[1]{\stackrel{#1}{=}}
\newcommand{\hash}{\ensuremath{\mathcal{H}}}
\newcommand{\negl}{{\tt neg}}
\newcommand{\cindist}{\stackrel{c}{\approx}}
\newcommand{\A}{{\mathcal{A}}}
\newcommand{\B}{{\mathcal{B}}}

\newcommand{\gen}{\mathsf{Gen}}
\newcommand{\keygen}{\mathsf{KeyGen}}
\newcommand{\RSA}{\mathsf{RSA}}
\newcommand{\pk}{\mathsf{pk}}
\newcommand{\PK}{\mathsf{PK}}
\newcommand{\SK}{\mathsf{SK}}
\newcommand{\CT}{\mathsf{CT}}
\newcommand{\sign}{\mathsf{Sign}}
\newcommand{\verify}{\mathsf{Verify}}
\newcommand{\enc}{\mathsf{Enc}}
\newcommand{\dec}{\mathsf{Dec}}
\newcommand{\mac}{\mathsf{Tag}}

\newcommand{\ASETUP}{\textsf{ABE.Setup}}
\newcommand{\AEXTRACT}{\textsf{ABE.Extract}}
\newcommand{\AENCRYPT}{\textsf{ABE.Enc}}
\newcommand{\ADECRYPT}{\textsf{ABE.Dec}}
\newcommand{\construction}{\mathit{Construction}}
\newcommand{\domain}{\mathit{Domain}}

\def\matA{\mathbf{A}}
\def\matT{\mathbf{T}}
\def\matB{\mathbf{B}}
\def\matF{\mathbf{F}}
\def\matH{\mathbf{H}}
\def\matM{\mathbf{M}}
\def\matS{\mathbf{S}}

\def\veca{\mathbf{a}}
\def\vecb{\mathbf{b}}
\def\vecc{\mathbf{c}}
\def\vecd{\mathbf{d}}
\def\vece{\mathbf{e}}
\def\vecu{\mathbf{u}}
\def\vecr{\mathbf{r}}
\def\vecm{{m}}
\def\vecs{\mathbf{s}}
\def\vect{{t}}
\def\vecv{{v}}
\def\vecw{\mathbf{w}}
\def\vecx{\mathbf{x}}
\def\vecy{\mathbf{y}}
\def\vecgamma{\mathbf{\gamma}}
\def\halfq{\left\lfloor\frac{q}{2}\right\rfloor}

\newcommand{\ZQ}{\mathbb{Z}_q}
\newcommand{\G}{\mathbb{G}}
\def\Q{\mathbb{Q}}
\newcommand{\nat}{\mathbb{N}}
\newcommand{\intg}{\mathbb{Z}}
\newcommand{\detm}{\mathsf{det}}

\newcommand{\getsr}{\overset{R}{\leftarrow}}

\setlength{\oddsidemargin}{.25in}
\setlength{\evensidemargin}{.25in}
\setlength{\textwidth}{6in}
\setlength{\topmargin}{-0.4in}
\setlength{\textheight}{8.5in}

\newcommand{\handout}[5]{
  % \renewcommand{\thepage}{#1-\arabic{page}}
   \noindent
   \begin{center}
   \framebox{
      \vbox{
    \hbox to 5.78in { {\bf CS60065: Cryptography and Network Security} \hfill {\it #2 }}
       \vspace{4mm}
       \hbox to 5.78in { {\Large \hfill #5  \hfill} }
       \vspace{2mm}
       \hbox to 5.78in { {\it #3 \hfill #4} }
      }
   }
   \end{center}
   \vspace*{4mm}
}

\newcommand{\ho}[4]{\handout{#1}{#2}{{\sf Instructor:}
#3}{}{Handout #1: #4}}

\newcommand{\lnotes}[3]{\handout{#1}{#2}{{\sf Instructor:}
#3}{}{Lecture #1}}


\newcommand{\quiz}[3]{\handout{#1}{#2}{{\sf Instructor:} #3}{Total: 40 pts}{Mid-Term #1}}

\newcommand{\major}[3]{\handout{#1}{#2}{{\sf Instructor:}
#3}{}{Major #1}}

\newcommand{\solution}[1]{#1}
%\homework{number}{out}{due}{instructor}
\newcommand{\homework}[4]{\handout{HW #1}{#2}{{\sf Instructor:} #4}{{\sf Due:} #3}{\textbf{Assignment #1}}}
\newcommand{\homeworksol}[4]{\handout{HW #1}{#2}{Name: #4}{{\sf Due:} #3}{Assignment #1}}

%================================================
% problemset macros
%================================================
% count problems
\newcounter{solutioncount}
\setcounter{solutioncount}{0}
\newcommand{\problem}[1]{%
\addtocounter{solutioncount}{1}%
\section*{Problem \arabic{solutioncount}: #1}}

% lets you make alphabetical lists (at first-level of enumeration)
\newenvironment
  {alphabetize}{\renewcommand{\theenumi}{\alph{enumi}}\begin{enumerate}}
  {\end{enumerate}\renewcommand{\theenumi}{\arabic{enumi}}}


\usepackage{amsfonts}
\usepackage{amsthm}
\usepackage{latexsym}
\usepackage{amsmath}
\usepackage{color}
\usepackage{comment}
\usepackage{enumitem}
\newcommand{\sk}{s}
\newtheorem{theorem}{Theorem}
\newtheorem{lesson}{Lesson}
\newtheorem{proposition}{Proposition}
\newtheorem{lemma}{Lemma}
\newtheorem{corollary}{Corollary}
\newtheorem{fact}{Fact}
\newtheorem*{claim}{Claim}
%\theoremstyle{definition}
\newtheorem{definition}{Definition}
\newtheorem{assumption}{Assumption}
\theoremstyle{remark}
\newtheorem{example}{Example}
\newtheorem*{remark}{Remark}

\DeclareSymbolFont{AMSb}{U}{msb}{m}{n}
\DeclareMathSymbol{\F}{\mathalpha}{AMSb}{"46}
\DeclareMathSymbol{\N}{\mathalpha}{AMSb}{"4E}
\DeclareMathSymbol{\R}{\mathalpha}{AMSb}{"52}
\DeclareMathSymbol{\X}{\mathalpha}{AMSb}{"58}
\DeclareMathSymbol{\Zz}{\mathalpha}{AMSb}{"5A}
\newcommand{\Z}[1]{{\ensuremath{\Zz_{#1}} }}
\newcommand{\Zs}[1]{\ensuremath{\Zz^{\ast}_{#1}}}
\newcommand{\Zn}{\Z{n}}
\newcommand{\Zns}{\Zs{n}}
\newcommand{\Zstar}[1]{\Zs{#1}}
\newcommand{\Zp}{{\Z{p}}}
\newcommand{\Zps}{\Zs{p}}
\newcommand{\Zqs}{\Zs{q}}
\newcommand{\Zq}{{\Z{q}}}
%\newcommand{\QR}{\mathop{\mathrm{QR}}\nolimits}
\newcommand{\ord}[1]{\mathop{\mathrm{ord}}({#1})}
\newcommand{\QR}[1]{\ensuremath{\textit{QR}_{#1}}}
\newcommand{\becomes}{:=}
\newcommand{\rem}[1]{\ensuremath{\ \operatorname{rem} #1}}  
\newcommand{\U}{{\mathcal{U}}}
\newcommand{\floor}[1]{\ensuremath{\lfloor{#1}\rfloor}}
\newcommand{\de}[1]{\ensuremath{\Delta{#1}}}
\newcommand{\js}[2]{\left( \frac{#1}{#2} \right)}

\renewcommand{\QR}{{\mbox{QR}}}
\newcommand{\QNR}{{\mbox{QNR}}}
\newcommand{\crt}{{\mbox{CRT}}}
\newcommand{\rsa}{{\mbox{RSA}}}
\newcommand{\rsamod}{{\mbox{RSA-modulus}}}


\newcommand{\greq}[1]{\stackrel{#1}{=}}
\newcommand{\hash}{\ensuremath{\mathcal{H}}}
\newcommand{\negl}{{\tt neg}}
\newcommand{\cindist}{\stackrel{c}{\approx}}
\newcommand{\A}{{\mathcal{A}}}
\newcommand{\B}{{\mathcal{B}}}

\newcommand{\gen}{\mathsf{Gen}}
\newcommand{\keygen}{\mathsf{KeyGen}}
\newcommand{\RSA}{\mathsf{RSA}}
\newcommand{\pk}{\mathsf{pk}}
\newcommand{\PK}{\mathsf{PK}}
\newcommand{\SK}{\mathsf{SK}}
\newcommand{\CT}{\mathsf{CT}}
\newcommand{\sign}{\mathsf{Sign}}
\newcommand{\verify}{\mathsf{Verify}}
\newcommand{\enc}{\mathsf{Enc}}
\newcommand{\dec}{\mathsf{Dec}}
\newcommand{\mac}{\mathsf{Tag}}

\newcommand{\ASETUP}{\textsf{ABE.Setup}}
\newcommand{\AEXTRACT}{\textsf{ABE.Extract}}
\newcommand{\AENCRYPT}{\textsf{ABE.Enc}}
\newcommand{\ADECRYPT}{\textsf{ABE.Dec}}
\newcommand{\construction}{\mathit{Construction}}
\newcommand{\domain}{\mathit{Domain}}

\def\matA{\mathbf{A}}
\def\matT{\mathbf{T}}
\def\matB{\mathbf{B}}
\def\matF{\mathbf{F}}
\def\matH{\mathbf{H}}
\def\matM{\mathbf{M}}
\def\matS{\mathbf{S}}

\def\veca{\mathbf{a}}
\def\vecb{\mathbf{b}}
\def\vecc{\mathbf{c}}
\def\vecd{\mathbf{d}}
\def\vece{\mathbf{e}}
\def\vecu{\mathbf{u}}
\def\vecr{\mathbf{r}}
\def\vecm{{m}}
\def\vecs{\mathbf{s}}
\def\vect{{t}}
\def\vecv{{v}}
\def\vecw{\mathbf{w}}
\def\vecx{\mathbf{x}}
\def\vecy{\mathbf{y}}
\def\vecgamma{\mathbf{\gamma}}
\def\halfq{\left\lfloor\frac{q}{2}\right\rfloor}

\newcommand{\ZQ}{\mathbb{Z}_q}
\newcommand{\G}{\mathbb{G}}
\def\Q{\mathbb{Q}}
\newcommand{\nat}{\mathbb{N}}
\newcommand{\intg}{\mathbb{Z}}
\newcommand{\detm}{\mathsf{det}}

\newcommand{\getsr}{\overset{R}{\leftarrow}}

\setlength{\oddsidemargin}{.25in}
\setlength{\evensidemargin}{.25in}
\setlength{\textwidth}{6in}
\setlength{\topmargin}{-0.4in}
\setlength{\textheight}{8.5in}

\newcommand{\handout}[5]{
  % \renewcommand{\thepage}{#1-\arabic{page}}
   \noindent
   \begin{center}
   \framebox{
      \vbox{
    \hbox to 5.78in { {\bf CS60065: Cryptography and Network Security} \hfill {\it #2 }}
       \vspace{4mm}
       \hbox to 5.78in { {\Large \hfill #5  \hfill} }
       \vspace{2mm}
       \hbox to 5.78in { {\it #3 \hfill #4} }
      }
   }
   \end{center}
   \vspace*{4mm}
}

\newcommand{\ho}[4]{\handout{#1}{#2}{{\sf Instructor:}
#3}{}{Handout #1: #4}}

\newcommand{\lnotes}[3]{\handout{#1}{#2}{{\sf Instructor:}
#3}{}{Lecture #1}}


\newcommand{\quiz}[3]{\handout{#1}{#2}{{\sf Instructor:} #3}{Total: 40 pts}{Mid-Term #1}}

\newcommand{\major}[3]{\handout{#1}{#2}{{\sf Instructor:}
#3}{}{Major #1}}

\newcommand{\solution}[1]{#1}
%\homework{number}{out}{due}{instructor}
\newcommand{\homework}[4]{\handout{HW #1}{#2}{{\sf Instructor:} #4}{{\sf Due:} #3}{\textbf{Assignment #1}}}
\newcommand{\homeworksol}[4]{\handout{HW #1}{#2}{Name: #4}{{\sf Due:} #3}{Assignment #1}}

%================================================
% problemset macros
%================================================
% count problems
\newcounter{solutioncount}
\setcounter{solutioncount}{0}
\newcommand{\problem}[1]{%
\addtocounter{solutioncount}{1}%
\section*{Problem \arabic{solutioncount}: #1}}

% lets you make alphabetical lists (at first-level of enumeration)
\newenvironment
  {alphabetize}{\renewcommand{\theenumi}{\alph{enumi}}\begin{enumerate}}
  {\end{enumerate}\renewcommand{\theenumi}{\arabic{enumi}}}

\usepackage{framed} %custom addition
\newcommand{\Mod}[1]{\ (\mathrm{mod}\ #1)}
\raggedbottom

\begin{document}
\raggedbottom
\homeworksol{4}{{\sf Total:} $50$ points}{11.59 pm, Nov 07, 2024}{Anit Mangal}
\vspace*{-0.4cm}
{ \sf \textbf{{\color{red}Note}}: {\color{blue} The basic policies are stated in the course page. Using GPT (or similar tools) to solve problems from the assignment is \textbf{strictly prohibited}. Use of any other (possibly online) source(s) \textbf{must} be clearly stated in the solution. \textbf{Any dishonesty, if caught, will yield zero credits for the entire assignment.}}}\\
\hrule
\begin{description}[leftmargin=3pt]
	\item[A. [RSA and Primality Tests : \!\!\!\! $\mathbf{10 + 5+5 = 20\ points.}$\!\!\!]] 
	\begin{framed}
		{\bf Answers}
		\begin{description}
			\item[A.1] We have $ed \equiv 1 \Mod{\lambda(n)}$\\
			\begin{flalign*}
				\lambda(n) &= \frac{\Phi(n)}{gcd(p-1,q-1)} &\\
				&= \frac{(p-1)(q-1)}{gcd(p-1,q-1)}&\\
				&= lcm(p-1,q-1) &[gcd(a,b)\times lcm(a,b) = a\times b]
			\end{flalign*}
			$\text{So, } \lambda(n) = k_1(p-1) \text{ and } \lambda(n) = k_2(q-1)$
			\begin{flalign*}
				\text{Since } &ed \equiv 1 \Mod{\lambda(n)},&\\
				& ed = t_1k_1(p-1) + 1 \text{ and } ed = t_2k_2(q-1) + 1&
			\end{flalign*}
			Now, decrypting ciphertext $c$,
			\begin{flalign*}
				c^d &\equiv \left(m^{e}\right)^d \Mod{n}&\\
				&\equiv m^{ed} \Mod{n}&
			\end{flalign*}
			We need to prove that $m^{ed} \equiv m \Mod{n}$. By Chinese Remainder Theorem, it suffices to show that $m^{ed} \equiv m \Mod{p}$ and $m^{ed} \equiv m \Mod{q}$.\\ \\
			Case 1: $m$ and $p$ are not co-prime.\\
			Since $p$ is prime, $m = rp$.\\
			So, $m \equiv 0\Mod{p}$ and $m^{ed} \equiv 0\Mod{p}$\\
			$\therefore m^{ed} \equiv m \Mod{p}$\\ \\
			Case 2: $m$ and $p$ are co-prime.
			\begin{flalign*}
				m^{ed}& \equiv m^{t_1k_1(p-1) + 1} \Mod{p}&\\
				&\equiv \left(m^{p-1}\right)^{t_1k_1} m \Mod{p}&\\
				&\equiv 1^{t_1k_1}m\Mod{p}&\text{[By Fermat's Theorem]}\\
				&\equiv m\Mod{p}&
			\end{flalign*}
			$\therefore m^{ed} \equiv m \Mod{p}$\\
			Similarly, $m^{ed} \equiv m \Mod{q}$\\
			{\color{blue} $\therefore m^{ed} \equiv m \Mod{n}$, which means encryption and decryption are still inverse operations.}
			\item[A.2a] We have $G(n) = \left\{a:a\in\Zstar{n}, \left(\frac{a}{n}\right)\equiv a^\frac{(n-1)}{2}\Mod{n}\right\}$\\
			First, we prove that $b\in G(n)\Rightarrow b^{-1} \in G(n)$.
			\begin{flalign*}
				\left(\frac{b}{n}\right)\left(\frac{b^{-1}}{n}\right) &= \left(\frac{b.b^{-1}}{n}\right)&\text{[Multiplicative Property of Jacobi symbol]}\\
				&= \left(\frac{1}{n}\right) &\\
				&= 1&\text{[Property of Jacobi Symbol]}
			\end{flalign*}
			Since Jacobi symbol can only take values -1, 0 and 1,
			\begin{flalign}
				&\Rightarrow \left(\frac{b}{n}\right) = \left(\frac{b^{-1}}{n}\right) &
			\end{flalign}
			\begin{flalign*}
				\text{Now, } b^{\frac{(n-1)}{2}}.\left(b^{-1}\right)^{\frac{(n-1)}{2}} &\equiv \left(b.b^{-1}\right)^{\frac{(n-1)}{2}}\Mod{n}&\\
				&\equiv 1^{\frac{(n-1)}{2}}\Mod{n}&\\
				&\equiv 1\Mod{n}&
			\end{flalign*}
			Also, $gcd(b,n) = 1 \Rightarrow gcd(b^{-1}, n)=1$\\
			So, $b^{n-1}\equiv 1\Mod{n}$ and $\left(b^{-1}\right)^{n-1}\equiv 1\Mod{n}$ (Fermat's Theorem)\\
			So, $b^{\frac{(n-1)}{2}}\equiv 1 \text{ or } -1\Mod{n}$. Same for $b^{-1}$
			\begin{flalign}
				&\Rightarrow b^{\frac{(n-1)}{2}} \equiv \left(b^{-1}\right)^{\frac{(n-1)}{2}} \Mod{n}&\\
				&\text{Also }\left(\frac{b}{n}\right)\equiv b^\frac{(n-1)}{2}\Mod{n}&
			\end{flalign}
			From (1), (2) and (3),
			\begin{flalign*}
				&\Rightarrow \left(\frac{b^{-1}}{n}\right)\equiv \left(\frac{b}{n}\right) \equiv b^{\frac{(n-1)}{2}} \equiv \left(b^{-1}\right)^\frac{(n-1)}{2}\Mod{n}&
			\end{flalign*}
			$\therefore b\in G(n)\Rightarrow b^{-1}\in G(n)$\\
			So, $\forall a, b \in G(n)$,\\
			$\left(\frac{a}{n}\right)\equiv a^{\frac{(n-1)}{2}}\Mod{n}$ and $\left(\frac{b^{-1}}{n}\right)\equiv \left(b^{-1}\right)^{\frac{(n-1)}{2}}\Mod{n}$\\
			$\Rightarrow \left(\frac{a.b^{-1}}{n}\right)\equiv \left(a.b^{-1}\right)^{\frac{(n-1)}{2}}\Mod{n}$\\
			$\therefore (a.b^{-1}) \in G(n)$\\
			So, $G(n)$ is a subgroup of $\Zstar{n}$.\\
			Using Lagrange's theorem, $|G(n)|$ divides $|\Zstar{n}|$.\\
			{\color{blue} $\therefore\ |G(n)| \leq \frac{|\Zstar{n}|}{2} \leq \frac{(n-1)}{2}$.}
			\clearpage
			\item[A.2b] We have $n = p^kq$ and $a = 1+p^{k-1}q$\\
			Let $t = \frac{(n-1)}{2}$. Since $n$ is odd (p and q are odd), $t \in \N$.
			\begin{flalign*}
				a^{\frac{(n-1)}{2}}&\equiv \left(1+p^{k-1}q\right)^t\Mod{n}&\\
				&\equiv 1 + {t\choose 1}(p^{k-1}q) + \dots + {t\choose t}\left(p^{k-1}q\right)^t\Mod{n}&\text{[Binomial Theorem]}\\
				&\equiv 1 + t(p^{k-1}q)\Mod{n}&[n | \left(p^{k-1}q\right)^t\ \forall t \geq 2]\\
				&\equiv 1 + (n-1)(2^{-1})(p^{k-1}q)\Mod{n}&\\
				&\equiv 1 - (p^{k-1}q)(2^{-1})\Mod{n}&\\
				&\equiv 1 + p^{k-1}q\Mod{n}&\\
				&\equiv a\Mod{n}&
			\end{flalign*}
			Now,
			\begin{flalign*}
				\left(\frac{a}{n}\right) &= \left(\frac{a}{p}\right)^k\left(\frac{a}{q}\right)&\text{[Property of Jacobi symbol]}\\
				&= \left(\frac{a}{q}\right)&[a \equiv 1\Mod{p}]\\
				&= 1&[a \equiv 1\Mod{q}]
			\end{flalign*}
			Since, $a = 1+p^{k-1}q \not\equiv 1\Mod{n}$,\\
			{\color{blue}$\therefore \left(\frac{a}{n}\right)\not\equiv a^{\frac{(n-1)}{2}}$.}
		\end{description}
	\end{framed}
	\clearpage
	\item[B. [ElGamal Encryption and Diffie-Hellman Problems: \!\!\!\! $\mathbf{10 + 10 = 20\ points.}$\!\!\!]] 
	\begin{framed}
		{\bf Answers}
		\begin{description}
			\item[B.1] Assume that the same ephemeral secret $k$ is used to encrypt $m_1$ and $m_2$ to ($c_{11},c_{12}$) and ($c_{21},c_{22}$) respectively, using generator $g$ on prime $p$, and the attacker knows the message $m_1$.
			\begin{flalign*}
				c_{12} &\equiv m_1 e^k \Mod{p}&\\
				&\equiv m_1 g^{dk}\Mod{p}&\\
				\Rightarrow c_{12} m_2 &\equiv m_1 m_2 g^{dk}\Mod{p}&\\
				\text{Also, } c_{22} &\equiv m_2 e^k \Mod{p}&\\
				&\equiv m_2 g^{dk}\Mod{p}&\\
				\Rightarrow c_{22} m_1 &\equiv m_1 m_2 g^{dk}\Mod{p}&
			\end{flalign*}
			So, $c_{12} m_2 \equiv c_{22} m_1\Mod{p}$ and hence, $m_2 \equiv c_{22}m_1{c_{12}}^{-1}\Mod{p}$.\\
			{\color{blue} $\therefore$ With $m_1$ known, the attacker can obtain $m_2$ if the same ephemeral secret is used to encrypt both of them.}
			\clearpage
			\item[B.2] We have $E_i \equiv g^{x_i}h_i^r\Mod{p}$ and $h_i = h^{t_i}g^{s_i}, \forall i \in [l]$
			\begin{flalign*}
				\Rightarrow E_i &\equiv g^{x_i}\left(h^{t_i}g^{s_i}\right)^r\Mod{p}&\\
				&\equiv g^{x_i}\left(h^{rt_i}g^{rs_i}\right)\Mod{p}&\\
				&\equiv g^{x_i + rs_i}h^{rt_i}\Mod{p}&
			\end{flalign*}
			We are given $p, C, D, \{E_1, E_2, \dots, E_l\}, \vec{\textbf{y}}, s_{\vec{\textbf{y}}}, t_{\vec{\textbf{y}}}$.\\
			Now, to decrypt, compute $m' := \left(\prod\limits_{i=1}^{l}E_{i}^{y_i}\right).\left(C^{s_{\vec{\textbf{y}}}}.D^{t_{\vec{\textbf{y}}}}\right)^{-1}\Mod{p}$
			\begin{flalign*}
				\left(\prod\limits_{i=1}^{l}E_{i}^{y_i}\right).\left(C^{s_{\vec{\textbf{y}}}}.D^{t_{\vec{\textbf{y}}}}\right)^{-1} &\equiv \left(\prod\limits_{i=1}^{l}\left(g^{x_i + rs_i}h^{rt_i}\right)^{y_i}\right).\left(\left(g^r\right)^{s_{\vec{\textbf{y}}}}.\left(h^r\right)^{t_{\vec{\textbf{y}}}}\right)^{-1}\Mod{p}&\\
				&\equiv \left(\prod\limits_{i=1}^{l}g^{x_iy_i + ry_is_i}h^{ry_it_i}\right).\left(g^{rs_{\vec{\textbf{y}}}}.h^{rt_{\vec{\textbf{y}}}}\right)^{-1}\Mod{p}&\\
				&\equiv\left(g^{\sum_{i=1}^{l}\left(x_iy_i + ry_is_i\right)}h^{\sum_{i=1}^{l}ry_it_i}\right).\left(g^{\sum_{i=1}^{l}rs_iy_i}.h^{\sum_{i=1}^{l}rt_iy_i}\right)^{-1}\Mod{p} &\\
				&\equiv g^{\sum_{i=1}^{l}\left(x_iy_i\right)}\left(g^{\sum_{i=1}^{l}ry_is_i}h^{\sum_{i=1}^{l}ry_it_i}\right).\left(g^{\sum_{i=1}^{l}rs_iy_i}.h^{\sum_{i=1}^{l}rt_iy_i}\right)^{-1}\Mod{p}&\\
				&\equiv g^{\sum_{i=1}^{l}\left(x_iy_i\right)}\Mod{p}&\\
				&\equiv g^{\langle\vec{\textbf{x}},\vec{\textbf{y}}\rangle}\Mod{p}&\\
			\end{flalign*}
			{\color{blue}
			So, $\left(\prod\limits_{i=1}^{l}E_{i}^{y_i}\right).\left(C^{s_{\vec{\textbf{y}}}}.D^{t_{\vec{\textbf{y}}}}\right)^{-1}$ yields $g^{\langle\vec{\textbf{x}},\vec{\textbf{y}}\rangle}\Mod{p}$}
		\end{description}
	\end{framed}
	\clearpage
	\item[C. [Elliptic Curves: \!\!\!\! $\mathbf{5 + 5 = 10\ points.}$\!\!\!]] 
	\begin{framed}
		{\bf Answers}
		\begin{description}
			\item[C.1] For curve $y^2 \equiv x^3 + 2x + 2$ over $\Z{17}$,
			\begin{flalign*}
				\text{Discriminant }\Delta &\equiv -16(4a^3+27b^2) \Mod{17}&(\text{For curve }y^2=x^3+ax+b)\\
				&\equiv -16.(4.{2}^3 + 27.{2}^2)\Mod{17}&\\
				&\equiv -16.(15 + 6)\Mod{17}&\\
				&\equiv -16.4\Mod{17}&\\
				&\equiv -13\Mod{17}&\\
				&\equiv 4\Mod{17}&
			\end{flalign*}
			{\color{blue}$\therefore$ Discriminant is $4\Mod{17}$.}\\
			$P = (13,7), Q = (6,3)$\\
			\begin{flalign*}
				\text{Slope }\lambda &\equiv (y_2-y_1).(x_2-x_1)^{-1}\Mod{17}& (\text{Line at points }(x_1,y_1),(x_2,y_2))\\
				&\equiv (3-7).(6-13)^{-1}\Mod{17}&\\
				&\equiv 13.(-7)^{-1}\Mod{17}&\\
				&\equiv 13.(10)^{-1}\Mod{17}&\\
				&\equiv 13.12\Mod{17}&(10.12 \equiv 1\Mod{17})\\
				&\equiv 3\Mod{17}&
			\end{flalign*}
			Equation of line through P and Q is $y\equiv 3.x + 3 - 3.6 \equiv 3.x + 2$ over $\Z{17}$\\
			Finding intersection point R($x_3,y_3$) of the line with the curve.
			\begin{flalign*}
				&(3x+2)^2 \equiv x^3+2x+2&\\
				\Rightarrow&9x^2 + 12x + 4\equiv x^3+2x+2&\\
				\Rightarrow&x^3+8x^2+7x+15\equiv 0&
			\end{flalign*}
			$\Rightarrow x_3 = -8-13-6\Mod{17} = 7\Mod{17}$\\
			Using equation of line, $y_3 = 3.7 + 2\Mod{17} = 6\Mod{17}$. The required point has $y = -6\Mod{17} = 11\Mod{17}$.\\
			{\color{blue}So, point $P+Q = (7,11)$.}
			\clearpage
			\item[C.2] $E: y^2 \equiv x^3 + 3x + 2$ over $\Z{7}$. Finding points
			\begin{flalign*}
				x = 0&\Rightarrow y^2 \equiv 2\Mod{7}\Rightarrow y=4\Mod{7} \text{ \& } y=3\Mod{7}&\\
				x = 1&\Rightarrow y^2 \equiv 6\Mod{7}\text{(No solution)}&\\
				x = 2&\Rightarrow y^2 \equiv 2\Mod{7}\Rightarrow y=4\Mod{7} \text{ \& } y=3\Mod{7}&\\
				x = 3&\Rightarrow y^2 \equiv 3\Mod{7}\text{(No solution)}&\\
				x = 4&\Rightarrow y^2 \equiv 1\Mod{7}\Rightarrow y=1\Mod{7} \text{ \& } y=6\Mod{7}&\\
				x = 5&\Rightarrow y^2 \equiv 2\Mod{7}\Rightarrow y=4\Mod{7} \text{ \& } y=3\Mod{7}&\\
				x = 6&\Rightarrow y^2 \equiv 5\Mod{7}\text{(No solution)}&
			\end{flalign*}
			{\color{blue} So, points on the curve are $(0,4),(0,3),(2,4),(2,3),(4,1),(4,6),(5,4),(5,3)$ and $\mathcal{O}$ (point in infinity).}\\
			The order of the group is 9.\\
			$\alpha = (0,3) \Rightarrow x_1 = 0, y_1 = 3$, Computing $2\alpha = (0,3) + (0,3)$
			\begin{flalign*}
				\lambda &= (3.0^2 + 3)(2.3)^{-1}\Mod{7}&\\
				&= 3.6^{-1}\Mod{7}&\\
				&= 3.6\Mod{7}&\\
				&= 4\Mod{7}
			\end{flalign*}
			So,
			\begin{flalign*}
				x_2 &= 4^2-0-0\Mod{7}&\\
				&=16\Mod{7}&\\
				&=2\Mod{7}&
			\end{flalign*}
			\begin{flalign*}
				y_2 &= 4(0-2)-3\Mod{7}&\\
				&= 3\Mod{7}&
			\end{flalign*}
			Hence, $2\alpha = (2,3)$. Calculating $3\alpha$.
			\begin{flalign*}
				\lambda &= (3.2^2 + 3)(2.3)^{-1} \Mod{7}&\\
				& = 6\Mod{7}&
			\end{flalign*}
			So, $x_3 = 6^2 - 2 - 2 = 4\Mod{7}$\\
			$y_3 = 6.(2-4)-3 = 6\Mod{7}$\\
			Hence, $3\alpha = (4,6)$. Calculating $4\alpha$.\\
			$\lambda = (3.4^2 + 3)(2.6)^{-1} = 6\Mod{7}$\\
			$x_4 = 6^2-4-4 = 0\Mod{7}$\\
			$y_4 = 4.(4-0)-6 = 3\Mod{7}$\\
			Hence, $4\alpha = (0,3) = \alpha$.\\
			{\color{blue} So, the order of $\alpha$ is 4 and $\alpha$ does not generate the group.}
		\end{description}
	\end{framed}
\end{description}
\end{document}
