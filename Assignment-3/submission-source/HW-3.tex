\documentclass{article}
\usepackage{amsmath}				% want AMS fonts
\usepackage{amssymb}                            % use AMS symbols
\usepackage{graphicx}
\usepackage{enumerate,url}
 %\usepackage[ruled]{algorithm2e}
%% preamble.tex
%% this should be included with a command like
%% %% preamble.tex
%% this should be included with a command like
%% %% preamble.tex
%% this should be included with a command like
%% \input{p}

\usepackage{amsfonts}
\usepackage{amsthm}
\usepackage{latexsym}
\usepackage{amsmath}
\usepackage{color}
\usepackage{comment}
\usepackage{enumitem}
\newcommand{\sk}{s}
\newtheorem{theorem}{Theorem}
\newtheorem{lesson}{Lesson}
\newtheorem{proposition}{Proposition}
\newtheorem{lemma}{Lemma}
\newtheorem{corollary}{Corollary}
\newtheorem{fact}{Fact}
\newtheorem*{claim}{Claim}
%\theoremstyle{definition}
\newtheorem{definition}{Definition}
\newtheorem{assumption}{Assumption}
\theoremstyle{remark}
\newtheorem{example}{Example}
\newtheorem*{remark}{Remark}

\DeclareSymbolFont{AMSb}{U}{msb}{m}{n}
\DeclareMathSymbol{\F}{\mathalpha}{AMSb}{"46}
\DeclareMathSymbol{\N}{\mathalpha}{AMSb}{"4E}
\DeclareMathSymbol{\R}{\mathalpha}{AMSb}{"52}
\DeclareMathSymbol{\X}{\mathalpha}{AMSb}{"58}
\DeclareMathSymbol{\Zz}{\mathalpha}{AMSb}{"5A}
\newcommand{\Z}[1]{{\ensuremath{\Zz_{#1}} }}
\newcommand{\Zs}[1]{\ensuremath{\Zz^{\ast}_{#1}}}
\newcommand{\Zn}{\Z{n}}
\newcommand{\Zns}{\Zs{n}}
\newcommand{\Zstar}[1]{\Zs{#1}}
\newcommand{\Zp}{{\Z{p}}}
\newcommand{\Zps}{\Zs{p}}
\newcommand{\Zqs}{\Zs{q}}
\newcommand{\Zq}{{\Z{q}}}
%\newcommand{\QR}{\mathop{\mathrm{QR}}\nolimits}
\newcommand{\ord}[1]{\mathop{\mathrm{ord}}({#1})}
\newcommand{\QR}[1]{\ensuremath{\textit{QR}_{#1}}}
\newcommand{\becomes}{:=}
\newcommand{\rem}[1]{\ensuremath{\ \operatorname{rem} #1}}  
\newcommand{\U}{{\mathcal{U}}}
\newcommand{\floor}[1]{\ensuremath{\lfloor{#1}\rfloor}}
\newcommand{\de}[1]{\ensuremath{\Delta{#1}}}
\newcommand{\js}[2]{\left( \frac{#1}{#2} \right)}

\renewcommand{\QR}{{\mbox{QR}}}
\newcommand{\QNR}{{\mbox{QNR}}}
\newcommand{\crt}{{\mbox{CRT}}}
\newcommand{\rsa}{{\mbox{RSA}}}
\newcommand{\rsamod}{{\mbox{RSA-modulus}}}


\newcommand{\greq}[1]{\stackrel{#1}{=}}
\newcommand{\hash}{\ensuremath{\mathcal{H}}}
\newcommand{\negl}{{\tt neg}}
\newcommand{\cindist}{\stackrel{c}{\approx}}
\newcommand{\A}{{\mathcal{A}}}
\newcommand{\B}{{\mathcal{B}}}

\newcommand{\gen}{\mathsf{Gen}}
\newcommand{\keygen}{\mathsf{KeyGen}}
\newcommand{\RSA}{\mathsf{RSA}}
\newcommand{\pk}{\mathsf{pk}}
\newcommand{\PK}{\mathsf{PK}}
\newcommand{\SK}{\mathsf{SK}}
\newcommand{\CT}{\mathsf{CT}}
\newcommand{\sign}{\mathsf{Sign}}
\newcommand{\verify}{\mathsf{Verify}}
\newcommand{\enc}{\mathsf{Enc}}
\newcommand{\dec}{\mathsf{Dec}}
\newcommand{\mac}{\mathsf{Tag}}

\newcommand{\ASETUP}{\textsf{ABE.Setup}}
\newcommand{\AEXTRACT}{\textsf{ABE.Extract}}
\newcommand{\AENCRYPT}{\textsf{ABE.Enc}}
\newcommand{\ADECRYPT}{\textsf{ABE.Dec}}
\newcommand{\construction}{\mathit{Construction}}
\newcommand{\domain}{\mathit{Domain}}

\def\matA{\mathbf{A}}
\def\matT{\mathbf{T}}
\def\matB{\mathbf{B}}
\def\matF{\mathbf{F}}
\def\matH{\mathbf{H}}
\def\matM{\mathbf{M}}
\def\matS{\mathbf{S}}

\def\veca{\mathbf{a}}
\def\vecb{\mathbf{b}}
\def\vecc{\mathbf{c}}
\def\vecd{\mathbf{d}}
\def\vece{\mathbf{e}}
\def\vecu{\mathbf{u}}
\def\vecr{\mathbf{r}}
\def\vecm{{m}}
\def\vecs{\mathbf{s}}
\def\vect{{t}}
\def\vecv{{v}}
\def\vecw{\mathbf{w}}
\def\vecx{\mathbf{x}}
\def\vecy{\mathbf{y}}
\def\vecgamma{\mathbf{\gamma}}
\def\halfq{\left\lfloor\frac{q}{2}\right\rfloor}

\newcommand{\ZQ}{\mathbb{Z}_q}
\newcommand{\G}{\mathbb{G}}
\def\Q{\mathbb{Q}}
\newcommand{\nat}{\mathbb{N}}
\newcommand{\intg}{\mathbb{Z}}
\newcommand{\detm}{\mathsf{det}}

\newcommand{\getsr}{\overset{R}{\leftarrow}}

\setlength{\oddsidemargin}{.25in}
\setlength{\evensidemargin}{.25in}
\setlength{\textwidth}{6in}
\setlength{\topmargin}{-0.4in}
\setlength{\textheight}{8.5in}

\newcommand{\handout}[5]{
  % \renewcommand{\thepage}{#1-\arabic{page}}
   \noindent
   \begin{center}
   \framebox{
      \vbox{
    \hbox to 5.78in { {\bf CS60065: Cryptography and Network Security} \hfill {\it #2 }}
       \vspace{4mm}
       \hbox to 5.78in { {\Large \hfill #5  \hfill} }
       \vspace{2mm}
       \hbox to 5.78in { {\it #3 \hfill #4} }
      }
   }
   \end{center}
   \vspace*{4mm}
}

\newcommand{\ho}[4]{\handout{#1}{#2}{{\sf Instructor:}
#3}{}{Handout #1: #4}}

\newcommand{\lnotes}[3]{\handout{#1}{#2}{{\sf Instructor:}
#3}{}{Lecture #1}}


\newcommand{\quiz}[3]{\handout{#1}{#2}{{\sf Instructor:} #3}{Total: 40 pts}{Mid-Term #1}}

\newcommand{\major}[3]{\handout{#1}{#2}{{\sf Instructor:}
#3}{}{Major #1}}

\newcommand{\solution}[1]{#1}
%\homework{number}{out}{due}{instructor}
\newcommand{\homework}[4]{\handout{HW #1}{#2}{{\sf Instructor:} #4}{{\sf Due:} #3}{\textbf{Assignment #1}}}
\newcommand{\homeworksol}[4]{\handout{HW #1}{#2}{Name: #4}{{\sf Due:} #3}{Assignment #1}}

%================================================
% problemset macros
%================================================
% count problems
\newcounter{solutioncount}
\setcounter{solutioncount}{0}
\newcommand{\problem}[1]{%
\addtocounter{solutioncount}{1}%
\section*{Problem \arabic{solutioncount}: #1}}

% lets you make alphabetical lists (at first-level of enumeration)
\newenvironment
  {alphabetize}{\renewcommand{\theenumi}{\alph{enumi}}\begin{enumerate}}
  {\end{enumerate}\renewcommand{\theenumi}{\arabic{enumi}}}


\usepackage{amsfonts}
\usepackage{amsthm}
\usepackage{latexsym}
\usepackage{amsmath}
\usepackage{color}
\usepackage{comment}
\usepackage{enumitem}
\newcommand{\sk}{s}
\newtheorem{theorem}{Theorem}
\newtheorem{lesson}{Lesson}
\newtheorem{proposition}{Proposition}
\newtheorem{lemma}{Lemma}
\newtheorem{corollary}{Corollary}
\newtheorem{fact}{Fact}
\newtheorem*{claim}{Claim}
%\theoremstyle{definition}
\newtheorem{definition}{Definition}
\newtheorem{assumption}{Assumption}
\theoremstyle{remark}
\newtheorem{example}{Example}
\newtheorem*{remark}{Remark}

\DeclareSymbolFont{AMSb}{U}{msb}{m}{n}
\DeclareMathSymbol{\F}{\mathalpha}{AMSb}{"46}
\DeclareMathSymbol{\N}{\mathalpha}{AMSb}{"4E}
\DeclareMathSymbol{\R}{\mathalpha}{AMSb}{"52}
\DeclareMathSymbol{\X}{\mathalpha}{AMSb}{"58}
\DeclareMathSymbol{\Zz}{\mathalpha}{AMSb}{"5A}
\newcommand{\Z}[1]{{\ensuremath{\Zz_{#1}} }}
\newcommand{\Zs}[1]{\ensuremath{\Zz^{\ast}_{#1}}}
\newcommand{\Zn}{\Z{n}}
\newcommand{\Zns}{\Zs{n}}
\newcommand{\Zstar}[1]{\Zs{#1}}
\newcommand{\Zp}{{\Z{p}}}
\newcommand{\Zps}{\Zs{p}}
\newcommand{\Zqs}{\Zs{q}}
\newcommand{\Zq}{{\Z{q}}}
%\newcommand{\QR}{\mathop{\mathrm{QR}}\nolimits}
\newcommand{\ord}[1]{\mathop{\mathrm{ord}}({#1})}
\newcommand{\QR}[1]{\ensuremath{\textit{QR}_{#1}}}
\newcommand{\becomes}{:=}
\newcommand{\rem}[1]{\ensuremath{\ \operatorname{rem} #1}}  
\newcommand{\U}{{\mathcal{U}}}
\newcommand{\floor}[1]{\ensuremath{\lfloor{#1}\rfloor}}
\newcommand{\de}[1]{\ensuremath{\Delta{#1}}}
\newcommand{\js}[2]{\left( \frac{#1}{#2} \right)}

\renewcommand{\QR}{{\mbox{QR}}}
\newcommand{\QNR}{{\mbox{QNR}}}
\newcommand{\crt}{{\mbox{CRT}}}
\newcommand{\rsa}{{\mbox{RSA}}}
\newcommand{\rsamod}{{\mbox{RSA-modulus}}}


\newcommand{\greq}[1]{\stackrel{#1}{=}}
\newcommand{\hash}{\ensuremath{\mathcal{H}}}
\newcommand{\negl}{{\tt neg}}
\newcommand{\cindist}{\stackrel{c}{\approx}}
\newcommand{\A}{{\mathcal{A}}}
\newcommand{\B}{{\mathcal{B}}}

\newcommand{\gen}{\mathsf{Gen}}
\newcommand{\keygen}{\mathsf{KeyGen}}
\newcommand{\RSA}{\mathsf{RSA}}
\newcommand{\pk}{\mathsf{pk}}
\newcommand{\PK}{\mathsf{PK}}
\newcommand{\SK}{\mathsf{SK}}
\newcommand{\CT}{\mathsf{CT}}
\newcommand{\sign}{\mathsf{Sign}}
\newcommand{\verify}{\mathsf{Verify}}
\newcommand{\enc}{\mathsf{Enc}}
\newcommand{\dec}{\mathsf{Dec}}
\newcommand{\mac}{\mathsf{Tag}}

\newcommand{\ASETUP}{\textsf{ABE.Setup}}
\newcommand{\AEXTRACT}{\textsf{ABE.Extract}}
\newcommand{\AENCRYPT}{\textsf{ABE.Enc}}
\newcommand{\ADECRYPT}{\textsf{ABE.Dec}}
\newcommand{\construction}{\mathit{Construction}}
\newcommand{\domain}{\mathit{Domain}}

\def\matA{\mathbf{A}}
\def\matT{\mathbf{T}}
\def\matB{\mathbf{B}}
\def\matF{\mathbf{F}}
\def\matH{\mathbf{H}}
\def\matM{\mathbf{M}}
\def\matS{\mathbf{S}}

\def\veca{\mathbf{a}}
\def\vecb{\mathbf{b}}
\def\vecc{\mathbf{c}}
\def\vecd{\mathbf{d}}
\def\vece{\mathbf{e}}
\def\vecu{\mathbf{u}}
\def\vecr{\mathbf{r}}
\def\vecm{{m}}
\def\vecs{\mathbf{s}}
\def\vect{{t}}
\def\vecv{{v}}
\def\vecw{\mathbf{w}}
\def\vecx{\mathbf{x}}
\def\vecy{\mathbf{y}}
\def\vecgamma{\mathbf{\gamma}}
\def\halfq{\left\lfloor\frac{q}{2}\right\rfloor}

\newcommand{\ZQ}{\mathbb{Z}_q}
\newcommand{\G}{\mathbb{G}}
\def\Q{\mathbb{Q}}
\newcommand{\nat}{\mathbb{N}}
\newcommand{\intg}{\mathbb{Z}}
\newcommand{\detm}{\mathsf{det}}

\newcommand{\getsr}{\overset{R}{\leftarrow}}

\setlength{\oddsidemargin}{.25in}
\setlength{\evensidemargin}{.25in}
\setlength{\textwidth}{6in}
\setlength{\topmargin}{-0.4in}
\setlength{\textheight}{8.5in}

\newcommand{\handout}[5]{
  % \renewcommand{\thepage}{#1-\arabic{page}}
   \noindent
   \begin{center}
   \framebox{
      \vbox{
    \hbox to 5.78in { {\bf CS60065: Cryptography and Network Security} \hfill {\it #2 }}
       \vspace{4mm}
       \hbox to 5.78in { {\Large \hfill #5  \hfill} }
       \vspace{2mm}
       \hbox to 5.78in { {\it #3 \hfill #4} }
      }
   }
   \end{center}
   \vspace*{4mm}
}

\newcommand{\ho}[4]{\handout{#1}{#2}{{\sf Instructor:}
#3}{}{Handout #1: #4}}

\newcommand{\lnotes}[3]{\handout{#1}{#2}{{\sf Instructor:}
#3}{}{Lecture #1}}


\newcommand{\quiz}[3]{\handout{#1}{#2}{{\sf Instructor:} #3}{Total: 40 pts}{Mid-Term #1}}

\newcommand{\major}[3]{\handout{#1}{#2}{{\sf Instructor:}
#3}{}{Major #1}}

\newcommand{\solution}[1]{#1}
%\homework{number}{out}{due}{instructor}
\newcommand{\homework}[4]{\handout{HW #1}{#2}{{\sf Instructor:} #4}{{\sf Due:} #3}{\textbf{Assignment #1}}}
\newcommand{\homeworksol}[4]{\handout{HW #1}{#2}{Name: #4}{{\sf Due:} #3}{Assignment #1}}

%================================================
% problemset macros
%================================================
% count problems
\newcounter{solutioncount}
\setcounter{solutioncount}{0}
\newcommand{\problem}[1]{%
\addtocounter{solutioncount}{1}%
\section*{Problem \arabic{solutioncount}: #1}}

% lets you make alphabetical lists (at first-level of enumeration)
\newenvironment
  {alphabetize}{\renewcommand{\theenumi}{\alph{enumi}}\begin{enumerate}}
  {\end{enumerate}\renewcommand{\theenumi}{\arabic{enumi}}}


\usepackage{amsfonts}
\usepackage{amsthm}
\usepackage{latexsym}
\usepackage{amsmath}
\usepackage{color}
\usepackage{comment}
\usepackage{enumitem}
\newcommand{\sk}{s}
\newtheorem{theorem}{Theorem}
\newtheorem{lesson}{Lesson}
\newtheorem{proposition}{Proposition}
\newtheorem{lemma}{Lemma}
\newtheorem{corollary}{Corollary}
\newtheorem{fact}{Fact}
\newtheorem*{claim}{Claim}
%\theoremstyle{definition}
\newtheorem{definition}{Definition}
\newtheorem{assumption}{Assumption}
\theoremstyle{remark}
\newtheorem{example}{Example}
\newtheorem*{remark}{Remark}

\DeclareSymbolFont{AMSb}{U}{msb}{m}{n}
\DeclareMathSymbol{\F}{\mathalpha}{AMSb}{"46}
\DeclareMathSymbol{\N}{\mathalpha}{AMSb}{"4E}
\DeclareMathSymbol{\R}{\mathalpha}{AMSb}{"52}
\DeclareMathSymbol{\X}{\mathalpha}{AMSb}{"58}
\DeclareMathSymbol{\Zz}{\mathalpha}{AMSb}{"5A}
\newcommand{\Z}[1]{{\ensuremath{\Zz_{#1}} }}
\newcommand{\Zs}[1]{\ensuremath{\Zz^{\ast}_{#1}}}
\newcommand{\Zn}{\Z{n}}
\newcommand{\Zns}{\Zs{n}}
\newcommand{\Zstar}[1]{\Zs{#1}}
\newcommand{\Zp}{{\Z{p}}}
\newcommand{\Zps}{\Zs{p}}
\newcommand{\Zqs}{\Zs{q}}
\newcommand{\Zq}{{\Z{q}}}
%\newcommand{\QR}{\mathop{\mathrm{QR}}\nolimits}
\newcommand{\ord}[1]{\mathop{\mathrm{ord}}({#1})}
\newcommand{\QR}[1]{\ensuremath{\textit{QR}_{#1}}}
\newcommand{\becomes}{:=}
\newcommand{\rem}[1]{\ensuremath{\ \operatorname{rem} #1}}  
\newcommand{\U}{{\mathcal{U}}}
\newcommand{\floor}[1]{\ensuremath{\lfloor{#1}\rfloor}}
\newcommand{\de}[1]{\ensuremath{\Delta{#1}}}
\newcommand{\js}[2]{\left( \frac{#1}{#2} \right)}

\renewcommand{\QR}{{\mbox{QR}}}
\newcommand{\QNR}{{\mbox{QNR}}}
\newcommand{\crt}{{\mbox{CRT}}}
\newcommand{\rsa}{{\mbox{RSA}}}
\newcommand{\rsamod}{{\mbox{RSA-modulus}}}


\newcommand{\greq}[1]{\stackrel{#1}{=}}
\newcommand{\hash}{\ensuremath{\mathcal{H}}}
\newcommand{\negl}{{\tt neg}}
\newcommand{\cindist}{\stackrel{c}{\approx}}
\newcommand{\A}{{\mathcal{A}}}
\newcommand{\B}{{\mathcal{B}}}

\newcommand{\gen}{\mathsf{Gen}}
\newcommand{\keygen}{\mathsf{KeyGen}}
\newcommand{\RSA}{\mathsf{RSA}}
\newcommand{\pk}{\mathsf{pk}}
\newcommand{\PK}{\mathsf{PK}}
\newcommand{\SK}{\mathsf{SK}}
\newcommand{\CT}{\mathsf{CT}}
\newcommand{\sign}{\mathsf{Sign}}
\newcommand{\verify}{\mathsf{Verify}}
\newcommand{\enc}{\mathsf{Enc}}
\newcommand{\dec}{\mathsf{Dec}}
\newcommand{\mac}{\mathsf{Tag}}

\newcommand{\ASETUP}{\textsf{ABE.Setup}}
\newcommand{\AEXTRACT}{\textsf{ABE.Extract}}
\newcommand{\AENCRYPT}{\textsf{ABE.Enc}}
\newcommand{\ADECRYPT}{\textsf{ABE.Dec}}
\newcommand{\construction}{\mathit{Construction}}
\newcommand{\domain}{\mathit{Domain}}

\def\matA{\mathbf{A}}
\def\matT{\mathbf{T}}
\def\matB{\mathbf{B}}
\def\matF{\mathbf{F}}
\def\matH{\mathbf{H}}
\def\matM{\mathbf{M}}
\def\matS{\mathbf{S}}

\def\veca{\mathbf{a}}
\def\vecb{\mathbf{b}}
\def\vecc{\mathbf{c}}
\def\vecd{\mathbf{d}}
\def\vece{\mathbf{e}}
\def\vecu{\mathbf{u}}
\def\vecr{\mathbf{r}}
\def\vecm{{m}}
\def\vecs{\mathbf{s}}
\def\vect{{t}}
\def\vecv{{v}}
\def\vecw{\mathbf{w}}
\def\vecx{\mathbf{x}}
\def\vecy{\mathbf{y}}
\def\vecgamma{\mathbf{\gamma}}
\def\halfq{\left\lfloor\frac{q}{2}\right\rfloor}

\newcommand{\ZQ}{\mathbb{Z}_q}
\newcommand{\G}{\mathbb{G}}
\def\Q{\mathbb{Q}}
\newcommand{\nat}{\mathbb{N}}
\newcommand{\intg}{\mathbb{Z}}
\newcommand{\detm}{\mathsf{det}}

\newcommand{\getsr}{\overset{R}{\leftarrow}}

\setlength{\oddsidemargin}{.25in}
\setlength{\evensidemargin}{.25in}
\setlength{\textwidth}{6in}
\setlength{\topmargin}{-0.4in}
\setlength{\textheight}{8.5in}

\newcommand{\handout}[5]{
  % \renewcommand{\thepage}{#1-\arabic{page}}
   \noindent
   \begin{center}
   \framebox{
      \vbox{
    \hbox to 5.78in { {\bf CS60065: Cryptography and Network Security} \hfill {\it #2 }}
       \vspace{4mm}
       \hbox to 5.78in { {\Large \hfill #5  \hfill} }
       \vspace{2mm}
       \hbox to 5.78in { {\it #3 \hfill #4} }
      }
   }
   \end{center}
   \vspace*{4mm}
}

\newcommand{\ho}[4]{\handout{#1}{#2}{{\sf Instructor:}
#3}{}{Handout #1: #4}}

\newcommand{\lnotes}[3]{\handout{#1}{#2}{{\sf Instructor:}
#3}{}{Lecture #1}}


\newcommand{\quiz}[3]{\handout{#1}{#2}{{\sf Instructor:} #3}{Total: 40 pts}{Mid-Term #1}}

\newcommand{\major}[3]{\handout{#1}{#2}{{\sf Instructor:}
#3}{}{Major #1}}

\newcommand{\solution}[1]{#1}
%\homework{number}{out}{due}{instructor}
\newcommand{\homework}[4]{\handout{HW #1}{#2}{{\sf Instructor:} #4}{{\sf Due:} #3}{\textbf{Assignment #1}}}
\newcommand{\homeworksol}[4]{\handout{HW #1}{#2}{Name: #4}{{\sf Due:} #3}{Assignment #1}}

%================================================
% problemset macros
%================================================
% count problems
\newcounter{solutioncount}
\setcounter{solutioncount}{0}
\newcommand{\problem}[1]{%
\addtocounter{solutioncount}{1}%
\section*{Problem \arabic{solutioncount}: #1}}

% lets you make alphabetical lists (at first-level of enumeration)
\newenvironment
  {alphabetize}{\renewcommand{\theenumi}{\alph{enumi}}\begin{enumerate}}
  {\end{enumerate}\renewcommand{\theenumi}{\arabic{enumi}}}

\usepackage{framed} %custom addition
\newcommand{\Mod}[1]{\ (\mathrm{mod}\ #1)}
\raggedbottom

\begin{document}
\raggedbottom
\homeworksol{3}{{\sf Total:} $50$ points}{11.59 pm, Oct 6, 2024}{Anit Mangal}
\vspace*{-0.4cm}
{ \sf \textbf{{\color{red}Note}}: {\color{blue} The basic policies are stated in the course page. Using GPT (or similar tools) to solve problems from the assignment is \textbf{strictly prohibited}. Use of any other (possibly online) source(s) \textbf{must} be clearly stated in the solution. \textbf{Any dishonesty, if caught, will yield zero credits for the entire assignment.}}}\\
\hrule
\begin{description}[leftmargin=3pt]
	\item[A. [Block Ciphers : \!\!\!\! $\mathbf{8 + 10 + (3\times 4) = 30\ points.}$\!\!\!]] 
	\begin{framed}
		{\bf Answers}
		\begin{description}
			\item[A.1] Bit representation for the S-box:
			\begin{center}
				\begin{tabular}{|c|c|c|c||c|c|c|c|}
					\hline $\mathbf{X_1}$ & $\mathbf{X_2}$ & $\mathbf{X_3}$ & $\mathbf{X_4}$ & $\mathbf{Y_1}$ & $\mathbf{Y_2}$ & $\mathbf{Y_3}$ & $\mathbf{Y_4}$ \\
					\hline 0 & 0 & 0 & 0 & 1 & 0 & 0 & 0 \\
					\hline 0 & 0 & 0 & 1 & 0 & 1 & 0 & 0 \\
					\hline 0 & 0 & 1 & 0 & 0 & 0 & 1 & 0 \\
					\hline 0 & 0 & 1 & 1 & 0 & 0 & 0 & 1 \\
					\hline 0 & 1 & 0 & 0 & 1 & 1 & 0 & 0 \\
					\hline 0 & 1 & 0 & 1 & 0 & 1 & 1 & 0 \\
					\hline 0 & 1 & 1 & 0 & 0 & 0 & 1 & 1 \\
					\hline 0 & 1 & 1 & 1 & 1 & 1 & 0 & 1 \\
					\hline 1 & 0 & 0 & 0 & 1 & 0 & 1 & 0 \\
					\hline 1 & 0 & 0 & 1 & 0 & 1 & 0 & 1 \\
					\hline 1 & 0 & 1 & 0 & 1 & 1 & 1 & 0 \\
					\hline 1 & 0 & 1 & 1 & 0 & 1 & 1 & 1 \\
					\hline 1 & 1 & 0 & 0 & 1 & 1 & 1 & 1 \\
					\hline 1 & 1 & 0 & 1 & 1 & 0 & 1 & 1 \\
					\hline 1 & 1 & 1 & 0 & 1 & 0 & 0 & 1 \\
					\hline 1 & 1 & 1 & 1 & 0 & 0 & 0 & 0 \\
					\hline
				\end{tabular}
			\end{center}
			\pagebreak
			\begin{flalign*}
				&\text{For an expression }
				\mathbf{E} = (\oplus_{i=1}^4\ a_i\mathbf{X_i})\oplus(\oplus_{j=1}^4\ b_j\mathbf{Y_j})&\\
				& \text{Writing count of $X,Y$ assignments which satisfy } \mathbf{E} = 0 \text{ in LAT}&
			\end{flalign*}
			\begin{center}
				\begin{tabular}{|c||c|c|c|c|c|c|c|c|c|c|c|c|c|c|c|c|}
					\hline & \multicolumn{15}{c}{$b$}&\\
					$a$ & 0 & 1 & 2 & 3 & 4 & 5 & 6 & 7 & 8 & 9 & $A$ & $B$ & $C$ & $D$ & $E$ & $F$ \\ \hline
					\hline 0 & 16 & 8 & 8 & 8 & 8 & 8 & 8 & 8 & 8 & 8 & 8 & 8 & 8 & 8 & 8 & 8\\
					\hline 1   & 8 & 10 & 6 & 8 & 10 & 8 & 8 & 6 & 4 & 6 & 6 & 8 & 10 & 8 & 4 & 10\\
					\hline 2   & 8 & 10 & 8 & 10 & 6 & 8 & 6 & 8 & 6 & 8 & 10 & 4 & 4 & 6 & 8 & 10\\
					\hline 3   & 8 & 8 & 10 & 10 & 8 & 12 & 10 & 6 & 6 & 6 & 8 & 8 & 10 & 6 & 12 & 8\\
					\hline 4   & 8 & 10 & 8 & 6 & 8 & 10 & 8 & 6 & 10 & 4 & 10 & 8 & 6 & 8 & 6 & 4\\
					\hline 5   & 8 & 12 & 6 & 6 & 10 & 10 & 8 & 12 & 6 & 12 & 8 & 8 & 8 & 8 & 10 & 6\\
					\hline 6   & 8 & 8 & 12 & 8 & 10 & 10 & 6 & 10 & 8 & 8 & 8 & 12 & 6 & 6 & 6 & 10\\
					\hline 7   & 8 & 6 & 6 & 8 & 12 & 6 & 10 & 8 & 8 & 6 & 6 & 8 & 4 & 6 & 10 & 8\\
					\hline 8   & 8 & 10 & 10 & 8 & 8 & 6 & 6 & 8 & 10 & 8 & 4 & 6 & 10 & 4 & 8 & 6\\
					\hline 9   & 8 & 8 & 8 & 12 & 10 & 10 & 6 & 10 & 10 & 6 & 6 & 6 & 8 & 12 & 8 & 8\\
					\hline $A$ & 8 & 12 & 10 & 10 & 6 & 6 & 12 & 8 & 8 & 8 & 6 & 10 & 6 & 10 & 8 & 8\\
					\hline $B$ & 8 & 6 & 12 & 6 & 8 & 6 & 8 & 10 & 4 & 6 & 8 & 6 & 8 & 10 & 8 & 6\\
					\hline $C$ & 8 & 8 & 10 & 10 & 12 & 8 & 10 & 6 & 8 & 12 & 10 & 6 & 8 & 8 & 6 & 6\\
					\hline $D$ & 8 & 6 & 8 & 6 & 6 & 12 & 10 & 8 & 8 & 10 & 4 & 6 & 6 & 8 & 6 & 8\\
					\hline $E$ & 8 & 6 & 6 & 12 & 6 & 8 & 8 & 10 & 6 & 8 & 8 & 10 & 8 & 6 & 6 & 4\\
					\hline $F$ & 8 & 8 & 8 & 8 & 8 & 8 & 12 & 12 & 10 & 6 & 10 & 6 & 10 & 6 & 6 & 10\\
					\hline
				\end{tabular}
			\end{center}
			\item[A.2] A brute-force attack against single DES takes $2^{56}$ DES encryption calls in the worst case, since the key is 56 bits long.\\
			Consider DESA, 
			\begin{equation*}
				y = \text{DESA}_{K,K'}(x) = \text{DES}_K(x)\oplus K'
			\end{equation*}
			If we have 2 valid message-ciphertext pairs $(x_1, y_1)$ and $(x_2, y_2)$ under DESA using key ($K$,$K'$),
			\begin{flalign*}
				y_1 \oplus y_2 &= (\text{DES}_K(x_1)\oplus K') \oplus (\text{DES}_K(x_2)\oplus K')\\
				&= \text{DES}_K(x_1) \oplus \text{DES}_K(x_2)
			\end{flalign*}
			We can thus brute-force on $K$ and verify by plugging it into the above equation. This requires $2^{57}$ DES encryption calls in the worst case.\\
			When we find $K$, we can find $K'$:
			\begin{flalign*}
				y_1 = \text{DES}_K(x_1)\oplus K'\\
				\Rightarrow K' = \text{DES}_K(x_1)\oplus y_1
			\end{flalign*}
			This requires one additional call to DES encryption.\\
			Hence, we would need roughly {\color{blue}$2^{57} + 1$ DES encryption calls} to break DESA encryption using a brute-force attack as compared to roughly {\color{blue}$2^{56}$ encryption calls} to break DES, making it around as difficult to break as DES.
			\item[A.3] Encrypting $x = 01101\ 11011\ 11010\ 00110$ using E($m_1m_2m_3m_4m_5$) = ($m_2m_5m_4m_1m_3$)
			\begin{enumerate}
				\item Mode ECB\\
				E(01101) = 11001\\
				E(11011) = 11110\\
				E(11010) = 10110\\
				E(00110) = 00101\\
				Hence, {\color{blue}E(01101 11011 11010 00110) = 11001 11110 10110 00101} with ECB mode
				\item Mode CBC\\
				IV = 11001, $x_1$ = 01101 $\Rightarrow$ $y_1$ = E(IV$\oplus x_1$) = E($11001\oplus01101$) = E(10100) = 00011\\
				$y_1$ = 00011, $x_2$ = 11011 $\Rightarrow$ $y_2$ = E($y_1\oplus x_2$) = E(11000) = 10010\\
				$y_2$ = 10010, $x_3$ = 11010 $\Rightarrow$ $y_3$ = E($y_2\oplus x_3$) = E(01000) = 10000\\
				$y_3$ = 10000, $x_4$ = 00110 $\Rightarrow$ $y_4$ = E($y_3\oplus x_4$) = E(10110) = 00111\\
				Hence, {\color{blue}E(01101 11011 11010 00110) = 00011 10010 10000 00111} with CBC mode
				\item Mode CFB\\
				IV = 11001, $x_1$ = 01101 $\Rightarrow$ $z_1$ = E(IV) = E(11001) = 11010 \\
				$\Rightarrow y_1 = z_1 \oplus x_1 = 11010\oplus 01101 = 10111$\\
				$x_2$ = 11011 $\Rightarrow$ $z_2$ = E($y_1$) = E(10111) = 01111\\
				$\Rightarrow y_2 = z_2 \oplus x_2 = 01111\oplus 11011 = 10100$\\
				$x_3$ = 11010 $\Rightarrow$ $z_3$ = E($y_2$) = E(10100) = 00011\\
				$\Rightarrow y_3 = z_3 \oplus x_3 = 00011\oplus 11010 = 11001$\\
				$x_4$ = 00110 $\Rightarrow$ $z_4$ = E($y_3$) = E(11001) = 11010\\
				$\Rightarrow y_4 = z_4 \oplus x_4 = 11010\oplus 00110 = 11100$\\
				Hence, {\color{blue}E(01101 11011 11010 00110) = 10111 10100 11001 11100} with CFB mode
				\item Mode OFB\\
				IV = 11001, $x_1$ = 01101 $\Rightarrow$ $z_1$ = E(IV) = E(11001) = 11010 \\
				$\Rightarrow y_1 = z_1 \oplus x_1 = 11010\oplus 01101 = 10111$\\
				$x_2$ = 11011 $\Rightarrow$ $z_2$ = E($z_1$) = E(11010) = 10110\\
				$\Rightarrow y_2 = z_2 \oplus x_2 = 10110\oplus 11011 = 01101$\\
				$x_3$ = 11010 $\Rightarrow$ $z_3$ = E($z_2$) = E(10110) = 00111\\
				$\Rightarrow y_3 = z_3 \oplus x_3 = 00111\oplus 11010 = 11101$\\
				$x_4$ = 00110 $\Rightarrow$ $z_4$ = E($z_3$) = E(00111) = 01101\\
				$\Rightarrow y_4 = z_4 \oplus x_4 = 01101\oplus 00110 = 01011$\\
				Hence, {\color{blue}E(01101 11011 11010 00110) = 10111 01101 11101 01011} with OFB mode
			\end{enumerate}
		\end{description}
	\end{framed}
	
	\item[B. [Cryptographic Hash Functions \& MACs: \!\!\!\!  $\mathbf{(3+4) + (4+3) + (3+3) = 20\ points.}$\!\!\!]] 	
	\begin{framed}
		{\bf Answers}
		\begin{description}
			\item[B.1] 
			\begin{enumerate}
				\item[a)]
				Failure probability = $\frac{\begin{array}{c}\text{Number of ways to choose $\mathcal{X}_0$}\\\text{$\mathcal{X}_0\bigcap h^{-1}(y) = \Phi$}\end{array}}{\begin{array}{c}\text{Number of ways to choose $\mathcal{X}_0$}\end{array}}$
				= $\frac{\left( \begin{array}{c}N-s_y \\ q\end{array}\right)}{\left( \begin{array}{c}N \\ q\end{array}\right)}$\\
				$\Rightarrow$ Success probability ($\epsilon$) = 1 - Failure probability =  1 - $\frac{\left( \begin{array}{c}N-s_y \\ q\end{array}\right)}{\left( \begin{array}{c}N \\ q\end{array}\right)}$
				\item[b)]
				$\epsilon_y = 1 - \frac{\left( \begin{array}{c}N-s_y \\ 1\end{array}\right)}{\left( \begin{array}{c}N \\ 1\end{array}\right)} = 1 - \frac{ \begin{array}{c}N-s_y\end{array}}{\begin{array}{c}N\end{array}} = \frac{\begin{array}{c}s_y\end{array}}{\begin{array}{c}N\end{array}}$\\
				\begin{flalign*}
				\text{Average success probability} &= \frac{1}{M}\cdot\sum_{y\in \mathcal{Y}}\epsilon_y&\\
				&=\frac{1}{M}\cdot\sum_{y\in \mathcal{Y}}\frac{s_y}{N} &\\
				&=\frac{1}{M\cdot N}\sum_{y\in \mathcal{Y}}{s_y}&\\
				&=\frac{1}{M\cdot N}\cdot N&\\
				&=\frac{1}{M}
				\end{flalign*}
				Hence, {\color{blue}Average success probability = $\frac{1}{M}$}
			\end{enumerate}
			\item[B.2]
			\begin{enumerate}
				\item[a)]
				Assume $x \ne x'$, let $h(x) = h(x')$
				\begin{enumerate}
					\item[Case 1.] $|x| = |x'| = n$
					\begin{flalign*}
						&h(x) = h(x')&\\
						\Rightarrow\ &0||x = 0||x'&\\
						\Rightarrow\ &x = x'&
					\end{flalign*}
					But we had assumed that $x \ne x'$. This is a contradiction. Hence, $h(x) \ne h(x')$
					\item[Case 2.] $|x| \ne n, |x'| \ne n$
					\begin{flalign*}
						&h(x) = h(x')&\\
						\Rightarrow\ &1||g(x) = 1||g(x')&\\
						\Rightarrow\ &g(x) = g(x')&\\
						\Rightarrow\ &x = x'&\text{($g$ is collision resistant)}
					\end{flalign*}
					But we had assumed that $x \ne x'$. This is a contradiction. Hence, $h(x) \ne h(x')$
					\item[Case 3.] $|x| = n, |x'| \ne n$ (Or $|x| \ne n, |x'| = n$)
					\begin{flalign*}
						&h(x) = h(x')&\\
						\Rightarrow\ &0||x = 1||g(x')&
					\end{flalign*}
					This is not possible since the first bit does not match. Hence, $h(x) \ne h(x')$
				\end{enumerate}
				Hence, {\color{blue}$h$ is collision-resistant if $g$ is collision-resistant.}
				\item[b)] Let $y$ be a message digest from $h(x)$.\\
				So, $|y| = n+1$.\\
				If $y_0 = 0$, we know that $y = 0 || x$, from definition of $h$\\
				$\Rightarrow\ x = y_1y_2\dots y_{n+1}$\\
				For all possible $n+1$-bit digests, half of the digests are such that $y_0 = 0$.\\
				Hence, {\color{blue}$h$ is not preimage-resistant.}
			\end{enumerate}
			\item[B.3] 
			\begin{enumerate}
				\item[a)]
				The MAC tag $x||h_K(x)$ is secure. This is because it does not allow forgery, and does not give information about K.
				\item[b)] MAC-and-Encrypt should be avoided. This is because:\\
				Suppose the secure block cipher scheme, CTR, is used. Mallory wants to forge a message $m$. It intercepts a cryptogram $E_K(m_0)||H_K(m_0)$, where $E_K(m_0)$ is of the form $IV||C_0$, $C_0$ is the size of $m_0$. Here, $m_0$ is known and its size assumed to be at least that of $m$.\\
				Mallory computes $K$ as $C_0\oplus m_0$, which can be truncated to size of $m$.\\
				Mallory computes $C = m\oplus K$.\\
				Mallory replaces the cryptogram by $IV||C$, causing a successful forgery.
			\end{enumerate}
		\end{description}
	\end{framed}
\end{description}
\end{document}
