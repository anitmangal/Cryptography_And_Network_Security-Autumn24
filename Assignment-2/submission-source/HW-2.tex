\documentclass{article}
\usepackage{amsmath}				% want AMS fonts
\usepackage{amssymb}                            % use AMS symbols
\usepackage{graphicx}
\usepackage{enumerate,url}
 %\usepackage[ruled]{algorithm2e}
%% preamble.tex
%% this should be included with a command like
%% %% preamble.tex
%% this should be included with a command like
%% %% preamble.tex
%% this should be included with a command like
%% \input{p}

\usepackage{amsfonts}
\usepackage{amsthm}
\usepackage{latexsym}
\usepackage{amsmath}
\usepackage{color}
\usepackage{comment}
\usepackage{enumitem}
\newcommand{\sk}{s}
\newtheorem{theorem}{Theorem}
\newtheorem{lesson}{Lesson}
\newtheorem{proposition}{Proposition}
\newtheorem{lemma}{Lemma}
\newtheorem{corollary}{Corollary}
\newtheorem{fact}{Fact}
\newtheorem*{claim}{Claim}
%\theoremstyle{definition}
\newtheorem{definition}{Definition}
\newtheorem{assumption}{Assumption}
\theoremstyle{remark}
\newtheorem{example}{Example}
\newtheorem*{remark}{Remark}

\DeclareSymbolFont{AMSb}{U}{msb}{m}{n}
\DeclareMathSymbol{\F}{\mathalpha}{AMSb}{"46}
\DeclareMathSymbol{\N}{\mathalpha}{AMSb}{"4E}
\DeclareMathSymbol{\R}{\mathalpha}{AMSb}{"52}
\DeclareMathSymbol{\X}{\mathalpha}{AMSb}{"58}
\DeclareMathSymbol{\Zz}{\mathalpha}{AMSb}{"5A}
\newcommand{\Z}[1]{{\ensuremath{\Zz_{#1}} }}
\newcommand{\Zs}[1]{\ensuremath{\Zz^{\ast}_{#1}}}
\newcommand{\Zn}{\Z{n}}
\newcommand{\Zns}{\Zs{n}}
\newcommand{\Zstar}[1]{\Zs{#1}}
\newcommand{\Zp}{{\Z{p}}}
\newcommand{\Zps}{\Zs{p}}
\newcommand{\Zqs}{\Zs{q}}
\newcommand{\Zq}{{\Z{q}}}
%\newcommand{\QR}{\mathop{\mathrm{QR}}\nolimits}
\newcommand{\ord}[1]{\mathop{\mathrm{ord}}({#1})}
\newcommand{\QR}[1]{\ensuremath{\textit{QR}_{#1}}}
\newcommand{\becomes}{:=}
\newcommand{\rem}[1]{\ensuremath{\ \operatorname{rem} #1}}  
\newcommand{\U}{{\mathcal{U}}}
\newcommand{\floor}[1]{\ensuremath{\lfloor{#1}\rfloor}}
\newcommand{\de}[1]{\ensuremath{\Delta{#1}}}
\newcommand{\js}[2]{\left( \frac{#1}{#2} \right)}

\renewcommand{\QR}{{\mbox{QR}}}
\newcommand{\QNR}{{\mbox{QNR}}}
\newcommand{\crt}{{\mbox{CRT}}}
\newcommand{\rsa}{{\mbox{RSA}}}
\newcommand{\rsamod}{{\mbox{RSA-modulus}}}


\newcommand{\greq}[1]{\stackrel{#1}{=}}
\newcommand{\hash}{\ensuremath{\mathcal{H}}}
\newcommand{\negl}{{\tt neg}}
\newcommand{\cindist}{\stackrel{c}{\approx}}
\newcommand{\A}{{\mathcal{A}}}
\newcommand{\B}{{\mathcal{B}}}

\newcommand{\gen}{\mathsf{Gen}}
\newcommand{\keygen}{\mathsf{KeyGen}}
\newcommand{\RSA}{\mathsf{RSA}}
\newcommand{\pk}{\mathsf{pk}}
\newcommand{\PK}{\mathsf{PK}}
\newcommand{\SK}{\mathsf{SK}}
\newcommand{\CT}{\mathsf{CT}}
\newcommand{\sign}{\mathsf{Sign}}
\newcommand{\verify}{\mathsf{Verify}}
\newcommand{\enc}{\mathsf{Enc}}
\newcommand{\dec}{\mathsf{Dec}}
\newcommand{\mac}{\mathsf{Tag}}

\newcommand{\ASETUP}{\textsf{ABE.Setup}}
\newcommand{\AEXTRACT}{\textsf{ABE.Extract}}
\newcommand{\AENCRYPT}{\textsf{ABE.Enc}}
\newcommand{\ADECRYPT}{\textsf{ABE.Dec}}
\newcommand{\construction}{\mathit{Construction}}
\newcommand{\domain}{\mathit{Domain}}

\def\matA{\mathbf{A}}
\def\matT{\mathbf{T}}
\def\matB{\mathbf{B}}
\def\matF{\mathbf{F}}
\def\matH{\mathbf{H}}
\def\matM{\mathbf{M}}
\def\matS{\mathbf{S}}

\def\veca{\mathbf{a}}
\def\vecb{\mathbf{b}}
\def\vecc{\mathbf{c}}
\def\vecd{\mathbf{d}}
\def\vece{\mathbf{e}}
\def\vecu{\mathbf{u}}
\def\vecr{\mathbf{r}}
\def\vecm{{m}}
\def\vecs{\mathbf{s}}
\def\vect{{t}}
\def\vecv{{v}}
\def\vecw{\mathbf{w}}
\def\vecx{\mathbf{x}}
\def\vecy{\mathbf{y}}
\def\vecgamma{\mathbf{\gamma}}
\def\halfq{\left\lfloor\frac{q}{2}\right\rfloor}

\newcommand{\ZQ}{\mathbb{Z}_q}
\newcommand{\G}{\mathbb{G}}
\def\Q{\mathbb{Q}}
\newcommand{\nat}{\mathbb{N}}
\newcommand{\intg}{\mathbb{Z}}
\newcommand{\detm}{\mathsf{det}}

\newcommand{\getsr}{\overset{R}{\leftarrow}}

\setlength{\oddsidemargin}{.25in}
\setlength{\evensidemargin}{.25in}
\setlength{\textwidth}{6in}
\setlength{\topmargin}{-0.4in}
\setlength{\textheight}{8.5in}

\newcommand{\handout}[5]{
  % \renewcommand{\thepage}{#1-\arabic{page}}
   \noindent
   \begin{center}
   \framebox{
      \vbox{
    \hbox to 5.78in { {\bf CS60065: Cryptography and Network Security} \hfill {\it #2 }}
       \vspace{4mm}
       \hbox to 5.78in { {\Large \hfill #5  \hfill} }
       \vspace{2mm}
       \hbox to 5.78in { {\it #3 \hfill #4} }
      }
   }
   \end{center}
   \vspace*{4mm}
}

\newcommand{\ho}[4]{\handout{#1}{#2}{{\sf Instructor:}
#3}{}{Handout #1: #4}}

\newcommand{\lnotes}[3]{\handout{#1}{#2}{{\sf Instructor:}
#3}{}{Lecture #1}}


\newcommand{\quiz}[3]{\handout{#1}{#2}{{\sf Instructor:} #3}{Total: 40 pts}{Mid-Term #1}}

\newcommand{\major}[3]{\handout{#1}{#2}{{\sf Instructor:}
#3}{}{Major #1}}

\newcommand{\solution}[1]{#1}
%\homework{number}{out}{due}{instructor}
\newcommand{\homework}[4]{\handout{HW #1}{#2}{{\sf Instructor:} #4}{{\sf Due:} #3}{\textbf{Assignment #1}}}
\newcommand{\homeworksol}[4]{\handout{HW #1}{#2}{Name: #4}{{\sf Due:} #3}{Assignment #1}}

%================================================
% problemset macros
%================================================
% count problems
\newcounter{solutioncount}
\setcounter{solutioncount}{0}
\newcommand{\problem}[1]{%
\addtocounter{solutioncount}{1}%
\section*{Problem \arabic{solutioncount}: #1}}

% lets you make alphabetical lists (at first-level of enumeration)
\newenvironment
  {alphabetize}{\renewcommand{\theenumi}{\alph{enumi}}\begin{enumerate}}
  {\end{enumerate}\renewcommand{\theenumi}{\arabic{enumi}}}


\usepackage{amsfonts}
\usepackage{amsthm}
\usepackage{latexsym}
\usepackage{amsmath}
\usepackage{color}
\usepackage{comment}
\usepackage{enumitem}
\newcommand{\sk}{s}
\newtheorem{theorem}{Theorem}
\newtheorem{lesson}{Lesson}
\newtheorem{proposition}{Proposition}
\newtheorem{lemma}{Lemma}
\newtheorem{corollary}{Corollary}
\newtheorem{fact}{Fact}
\newtheorem*{claim}{Claim}
%\theoremstyle{definition}
\newtheorem{definition}{Definition}
\newtheorem{assumption}{Assumption}
\theoremstyle{remark}
\newtheorem{example}{Example}
\newtheorem*{remark}{Remark}

\DeclareSymbolFont{AMSb}{U}{msb}{m}{n}
\DeclareMathSymbol{\F}{\mathalpha}{AMSb}{"46}
\DeclareMathSymbol{\N}{\mathalpha}{AMSb}{"4E}
\DeclareMathSymbol{\R}{\mathalpha}{AMSb}{"52}
\DeclareMathSymbol{\X}{\mathalpha}{AMSb}{"58}
\DeclareMathSymbol{\Zz}{\mathalpha}{AMSb}{"5A}
\newcommand{\Z}[1]{{\ensuremath{\Zz_{#1}} }}
\newcommand{\Zs}[1]{\ensuremath{\Zz^{\ast}_{#1}}}
\newcommand{\Zn}{\Z{n}}
\newcommand{\Zns}{\Zs{n}}
\newcommand{\Zstar}[1]{\Zs{#1}}
\newcommand{\Zp}{{\Z{p}}}
\newcommand{\Zps}{\Zs{p}}
\newcommand{\Zqs}{\Zs{q}}
\newcommand{\Zq}{{\Z{q}}}
%\newcommand{\QR}{\mathop{\mathrm{QR}}\nolimits}
\newcommand{\ord}[1]{\mathop{\mathrm{ord}}({#1})}
\newcommand{\QR}[1]{\ensuremath{\textit{QR}_{#1}}}
\newcommand{\becomes}{:=}
\newcommand{\rem}[1]{\ensuremath{\ \operatorname{rem} #1}}  
\newcommand{\U}{{\mathcal{U}}}
\newcommand{\floor}[1]{\ensuremath{\lfloor{#1}\rfloor}}
\newcommand{\de}[1]{\ensuremath{\Delta{#1}}}
\newcommand{\js}[2]{\left( \frac{#1}{#2} \right)}

\renewcommand{\QR}{{\mbox{QR}}}
\newcommand{\QNR}{{\mbox{QNR}}}
\newcommand{\crt}{{\mbox{CRT}}}
\newcommand{\rsa}{{\mbox{RSA}}}
\newcommand{\rsamod}{{\mbox{RSA-modulus}}}


\newcommand{\greq}[1]{\stackrel{#1}{=}}
\newcommand{\hash}{\ensuremath{\mathcal{H}}}
\newcommand{\negl}{{\tt neg}}
\newcommand{\cindist}{\stackrel{c}{\approx}}
\newcommand{\A}{{\mathcal{A}}}
\newcommand{\B}{{\mathcal{B}}}

\newcommand{\gen}{\mathsf{Gen}}
\newcommand{\keygen}{\mathsf{KeyGen}}
\newcommand{\RSA}{\mathsf{RSA}}
\newcommand{\pk}{\mathsf{pk}}
\newcommand{\PK}{\mathsf{PK}}
\newcommand{\SK}{\mathsf{SK}}
\newcommand{\CT}{\mathsf{CT}}
\newcommand{\sign}{\mathsf{Sign}}
\newcommand{\verify}{\mathsf{Verify}}
\newcommand{\enc}{\mathsf{Enc}}
\newcommand{\dec}{\mathsf{Dec}}
\newcommand{\mac}{\mathsf{Tag}}

\newcommand{\ASETUP}{\textsf{ABE.Setup}}
\newcommand{\AEXTRACT}{\textsf{ABE.Extract}}
\newcommand{\AENCRYPT}{\textsf{ABE.Enc}}
\newcommand{\ADECRYPT}{\textsf{ABE.Dec}}
\newcommand{\construction}{\mathit{Construction}}
\newcommand{\domain}{\mathit{Domain}}

\def\matA{\mathbf{A}}
\def\matT{\mathbf{T}}
\def\matB{\mathbf{B}}
\def\matF{\mathbf{F}}
\def\matH{\mathbf{H}}
\def\matM{\mathbf{M}}
\def\matS{\mathbf{S}}

\def\veca{\mathbf{a}}
\def\vecb{\mathbf{b}}
\def\vecc{\mathbf{c}}
\def\vecd{\mathbf{d}}
\def\vece{\mathbf{e}}
\def\vecu{\mathbf{u}}
\def\vecr{\mathbf{r}}
\def\vecm{{m}}
\def\vecs{\mathbf{s}}
\def\vect{{t}}
\def\vecv{{v}}
\def\vecw{\mathbf{w}}
\def\vecx{\mathbf{x}}
\def\vecy{\mathbf{y}}
\def\vecgamma{\mathbf{\gamma}}
\def\halfq{\left\lfloor\frac{q}{2}\right\rfloor}

\newcommand{\ZQ}{\mathbb{Z}_q}
\newcommand{\G}{\mathbb{G}}
\def\Q{\mathbb{Q}}
\newcommand{\nat}{\mathbb{N}}
\newcommand{\intg}{\mathbb{Z}}
\newcommand{\detm}{\mathsf{det}}

\newcommand{\getsr}{\overset{R}{\leftarrow}}

\setlength{\oddsidemargin}{.25in}
\setlength{\evensidemargin}{.25in}
\setlength{\textwidth}{6in}
\setlength{\topmargin}{-0.4in}
\setlength{\textheight}{8.5in}

\newcommand{\handout}[5]{
  % \renewcommand{\thepage}{#1-\arabic{page}}
   \noindent
   \begin{center}
   \framebox{
      \vbox{
    \hbox to 5.78in { {\bf CS60065: Cryptography and Network Security} \hfill {\it #2 }}
       \vspace{4mm}
       \hbox to 5.78in { {\Large \hfill #5  \hfill} }
       \vspace{2mm}
       \hbox to 5.78in { {\it #3 \hfill #4} }
      }
   }
   \end{center}
   \vspace*{4mm}
}

\newcommand{\ho}[4]{\handout{#1}{#2}{{\sf Instructor:}
#3}{}{Handout #1: #4}}

\newcommand{\lnotes}[3]{\handout{#1}{#2}{{\sf Instructor:}
#3}{}{Lecture #1}}


\newcommand{\quiz}[3]{\handout{#1}{#2}{{\sf Instructor:} #3}{Total: 40 pts}{Mid-Term #1}}

\newcommand{\major}[3]{\handout{#1}{#2}{{\sf Instructor:}
#3}{}{Major #1}}

\newcommand{\solution}[1]{#1}
%\homework{number}{out}{due}{instructor}
\newcommand{\homework}[4]{\handout{HW #1}{#2}{{\sf Instructor:} #4}{{\sf Due:} #3}{\textbf{Assignment #1}}}
\newcommand{\homeworksol}[4]{\handout{HW #1}{#2}{Name: #4}{{\sf Due:} #3}{Assignment #1}}

%================================================
% problemset macros
%================================================
% count problems
\newcounter{solutioncount}
\setcounter{solutioncount}{0}
\newcommand{\problem}[1]{%
\addtocounter{solutioncount}{1}%
\section*{Problem \arabic{solutioncount}: #1}}

% lets you make alphabetical lists (at first-level of enumeration)
\newenvironment
  {alphabetize}{\renewcommand{\theenumi}{\alph{enumi}}\begin{enumerate}}
  {\end{enumerate}\renewcommand{\theenumi}{\arabic{enumi}}}


\usepackage{amsfonts}
\usepackage{amsthm}
\usepackage{latexsym}
\usepackage{amsmath}
\usepackage{color}
\usepackage{comment}
\usepackage{enumitem}
\newcommand{\sk}{s}
\newtheorem{theorem}{Theorem}
\newtheorem{lesson}{Lesson}
\newtheorem{proposition}{Proposition}
\newtheorem{lemma}{Lemma}
\newtheorem{corollary}{Corollary}
\newtheorem{fact}{Fact}
\newtheorem*{claim}{Claim}
%\theoremstyle{definition}
\newtheorem{definition}{Definition}
\newtheorem{assumption}{Assumption}
\theoremstyle{remark}
\newtheorem{example}{Example}
\newtheorem*{remark}{Remark}

\DeclareSymbolFont{AMSb}{U}{msb}{m}{n}
\DeclareMathSymbol{\F}{\mathalpha}{AMSb}{"46}
\DeclareMathSymbol{\N}{\mathalpha}{AMSb}{"4E}
\DeclareMathSymbol{\R}{\mathalpha}{AMSb}{"52}
\DeclareMathSymbol{\X}{\mathalpha}{AMSb}{"58}
\DeclareMathSymbol{\Zz}{\mathalpha}{AMSb}{"5A}
\newcommand{\Z}[1]{{\ensuremath{\Zz_{#1}} }}
\newcommand{\Zs}[1]{\ensuremath{\Zz^{\ast}_{#1}}}
\newcommand{\Zn}{\Z{n}}
\newcommand{\Zns}{\Zs{n}}
\newcommand{\Zstar}[1]{\Zs{#1}}
\newcommand{\Zp}{{\Z{p}}}
\newcommand{\Zps}{\Zs{p}}
\newcommand{\Zqs}{\Zs{q}}
\newcommand{\Zq}{{\Z{q}}}
%\newcommand{\QR}{\mathop{\mathrm{QR}}\nolimits}
\newcommand{\ord}[1]{\mathop{\mathrm{ord}}({#1})}
\newcommand{\QR}[1]{\ensuremath{\textit{QR}_{#1}}}
\newcommand{\becomes}{:=}
\newcommand{\rem}[1]{\ensuremath{\ \operatorname{rem} #1}}  
\newcommand{\U}{{\mathcal{U}}}
\newcommand{\floor}[1]{\ensuremath{\lfloor{#1}\rfloor}}
\newcommand{\de}[1]{\ensuremath{\Delta{#1}}}
\newcommand{\js}[2]{\left( \frac{#1}{#2} \right)}

\renewcommand{\QR}{{\mbox{QR}}}
\newcommand{\QNR}{{\mbox{QNR}}}
\newcommand{\crt}{{\mbox{CRT}}}
\newcommand{\rsa}{{\mbox{RSA}}}
\newcommand{\rsamod}{{\mbox{RSA-modulus}}}


\newcommand{\greq}[1]{\stackrel{#1}{=}}
\newcommand{\hash}{\ensuremath{\mathcal{H}}}
\newcommand{\negl}{{\tt neg}}
\newcommand{\cindist}{\stackrel{c}{\approx}}
\newcommand{\A}{{\mathcal{A}}}
\newcommand{\B}{{\mathcal{B}}}

\newcommand{\gen}{\mathsf{Gen}}
\newcommand{\keygen}{\mathsf{KeyGen}}
\newcommand{\RSA}{\mathsf{RSA}}
\newcommand{\pk}{\mathsf{pk}}
\newcommand{\PK}{\mathsf{PK}}
\newcommand{\SK}{\mathsf{SK}}
\newcommand{\CT}{\mathsf{CT}}
\newcommand{\sign}{\mathsf{Sign}}
\newcommand{\verify}{\mathsf{Verify}}
\newcommand{\enc}{\mathsf{Enc}}
\newcommand{\dec}{\mathsf{Dec}}
\newcommand{\mac}{\mathsf{Tag}}

\newcommand{\ASETUP}{\textsf{ABE.Setup}}
\newcommand{\AEXTRACT}{\textsf{ABE.Extract}}
\newcommand{\AENCRYPT}{\textsf{ABE.Enc}}
\newcommand{\ADECRYPT}{\textsf{ABE.Dec}}
\newcommand{\construction}{\mathit{Construction}}
\newcommand{\domain}{\mathit{Domain}}

\def\matA{\mathbf{A}}
\def\matT{\mathbf{T}}
\def\matB{\mathbf{B}}
\def\matF{\mathbf{F}}
\def\matH{\mathbf{H}}
\def\matM{\mathbf{M}}
\def\matS{\mathbf{S}}

\def\veca{\mathbf{a}}
\def\vecb{\mathbf{b}}
\def\vecc{\mathbf{c}}
\def\vecd{\mathbf{d}}
\def\vece{\mathbf{e}}
\def\vecu{\mathbf{u}}
\def\vecr{\mathbf{r}}
\def\vecm{{m}}
\def\vecs{\mathbf{s}}
\def\vect{{t}}
\def\vecv{{v}}
\def\vecw{\mathbf{w}}
\def\vecx{\mathbf{x}}
\def\vecy{\mathbf{y}}
\def\vecgamma{\mathbf{\gamma}}
\def\halfq{\left\lfloor\frac{q}{2}\right\rfloor}

\newcommand{\ZQ}{\mathbb{Z}_q}
\newcommand{\G}{\mathbb{G}}
\def\Q{\mathbb{Q}}
\newcommand{\nat}{\mathbb{N}}
\newcommand{\intg}{\mathbb{Z}}
\newcommand{\detm}{\mathsf{det}}

\newcommand{\getsr}{\overset{R}{\leftarrow}}

\setlength{\oddsidemargin}{.25in}
\setlength{\evensidemargin}{.25in}
\setlength{\textwidth}{6in}
\setlength{\topmargin}{-0.4in}
\setlength{\textheight}{8.5in}

\newcommand{\handout}[5]{
  % \renewcommand{\thepage}{#1-\arabic{page}}
   \noindent
   \begin{center}
   \framebox{
      \vbox{
    \hbox to 5.78in { {\bf CS60065: Cryptography and Network Security} \hfill {\it #2 }}
       \vspace{4mm}
       \hbox to 5.78in { {\Large \hfill #5  \hfill} }
       \vspace{2mm}
       \hbox to 5.78in { {\it #3 \hfill #4} }
      }
   }
   \end{center}
   \vspace*{4mm}
}

\newcommand{\ho}[4]{\handout{#1}{#2}{{\sf Instructor:}
#3}{}{Handout #1: #4}}

\newcommand{\lnotes}[3]{\handout{#1}{#2}{{\sf Instructor:}
#3}{}{Lecture #1}}


\newcommand{\quiz}[3]{\handout{#1}{#2}{{\sf Instructor:} #3}{Total: 40 pts}{Mid-Term #1}}

\newcommand{\major}[3]{\handout{#1}{#2}{{\sf Instructor:}
#3}{}{Major #1}}

\newcommand{\solution}[1]{#1}
%\homework{number}{out}{due}{instructor}
\newcommand{\homework}[4]{\handout{HW #1}{#2}{{\sf Instructor:} #4}{{\sf Due:} #3}{\textbf{Assignment #1}}}
\newcommand{\homeworksol}[4]{\handout{HW #1}{#2}{Name: #4}{{\sf Due:} #3}{Assignment #1}}

%================================================
% problemset macros
%================================================
% count problems
\newcounter{solutioncount}
\setcounter{solutioncount}{0}
\newcommand{\problem}[1]{%
\addtocounter{solutioncount}{1}%
\section*{Problem \arabic{solutioncount}: #1}}

% lets you make alphabetical lists (at first-level of enumeration)
\newenvironment
  {alphabetize}{\renewcommand{\theenumi}{\alph{enumi}}\begin{enumerate}}
  {\end{enumerate}\renewcommand{\theenumi}{\arabic{enumi}}}

\usepackage{framed} %custom addition
\newcommand{\Mod}[1]{\ (\mathrm{mod}\ #1)}
\raggedbottom

\begin{document}
\raggedbottom
\homeworksol{2}{{\sf Total:} $40$ points}{11.59 pm, Aug 20, 2024}{Anit Mangal}
\vspace*{-0.4cm}
{ \sf \textbf{{\color{red}Note}}: {\color{blue} The basic policies are stated in the course page. Using GPT (or similar tools) to solve problems from the assignment is \textbf{strictly prohibited}. Use of any other (possibly online) source(s) \textbf{must} be clearly stated in the solution. \textbf{Any dishonesty, if caught, will yield zero credits for the entire assignment.}}}\\\vspace*{-0.2cm}
\hrule
\begin{description}[leftmargin=*]
	\item[A. [Perfect Secrecy : \!\!\!\!  $\mathbf{4 \times 4 = 16\ points.}$\!\!\!]]
	
	\begin{framed}
		{\bf Answers}
		\begin{description}
			\item[A.1] 
			Assuming perfect secrecy holds if for \textbf{any} probability distribution over $\mathbf{M}$ and $\mathbf{K}$, every ciphertext $\mathsf{C} \in \mathbf{C}$ and every message $\mathsf{M_1, M_2} \in \mathbf{M}$:
			\begin{flalign}
				\Pr[M_R = \mathsf{M_1} | C_R = \mathsf{C}] = \Pr[M_R = \mathsf{M_2} | C_R = \mathsf{C}]
			\end{flalign}
			Since perfect secrecy holds,
			\begin{align*}
				\Pr[M_R = \mathsf{M_1} | C_R = \mathsf{C}] &= \Pr[M_R = \mathsf{M_1}]\\
				\Rightarrow \Pr[M_R = \mathsf{M_1}] &= \Pr[M_R = \mathsf{M_2}] \text{\qquad\qquad(From (1))}
			\end{align*}
			But, our assumption is that perfect secrecy would hold for any probability distribution. This contradicts our assumption. So, our assumption is wrong.\\
			{\color{blue}Hence, $\Pr[M_R = \mathsf{M_1} | C_R = \mathsf{C}] = \Pr[M_R = \mathsf{M_2} | C_R = \mathsf{C}]$ does not imply perfect secrecy.}
			\item[A.2] 
			$\forall \ \mathsf{C_1},\mathsf{C_2}\in\mathbf{C}$
			\begin{flalign*}
				\Pr[\mathbf{C}_R = \mathsf{C_1}] &= \sum_{\mathsf{K}\in\mathbf{K}}\Pr[\mathbf{K}_R = \mathsf{K}]\Pr[\mathbf{M}_R = D_\mathsf{K}(\mathsf{C_1})]&\text{(Total Probability)}&\\
				&=\sum_{\mathsf{K}\in\mathbf{K}}\frac{1}{|\mathbf{K}|}\Pr[\mathbf{M}_R = D_\mathsf{K}(\mathsf{C_1})]&\text{(Using Shannon's Theorem)}&\\
				&=\frac{1}{|\mathbf{K}|}\sum_{\mathsf{K}\in\mathbf{K}}\Pr[\mathbf{M}_R = D_\mathsf{K}(\mathsf{C_1})]&
				\intertext{Using Shannon's Theorem, for every $\mathsf{M}\in\mathbf{M}$ and for every $\mathsf{C}\in\mathbf{C}$, there is a unique key $\mathsf{K}$ such that $E_\mathsf{K}(\mathsf{M}) = \mathsf{C}$. So, for every $\mathsf{K}\in\mathbf{K}$ and for every $\mathsf{C}\in\mathbf{C}$, there is a unique plaintext $\mathsf{M}$ such that $E_\mathsf{K}(\mathsf{M}) = \mathsf{C}$. And since $|\mathbf{K}| = |\mathbf{M}|$, every $\mathsf{M}$ is utilised if every $\mathsf{K}$ is utilised.}
				&=\frac{1}{|\mathbf{K}|}\sum_{\mathsf{M}\in\mathbf{M}}\Pr[\mathbf{M}_R = \mathsf{M}]&\\
				&=\frac{1}{|\mathbf{K}|}&\\
			\end{flalign*}
			Similiarly,
			\begin{flalign*}
				&\Pr[\mathbf{C}_R = \mathsf{C_2}] =\frac{1}{|\mathbf{K}|} = \Pr[\mathbf{C}_R = \mathsf{C_1}]&
			\end{flalign*}
			{\color{blue}$\therefore$ Every ciphertext is  equally probable.}
			\item[A.3] For Affine Cipher, $\mathsf{K}=(a,b): gcd(a, 26) = 1$\\
			For any  $y \in \Z{26}$,
			\begin{flalign*}
				\Pr[\mathbf{C}_R = y] &= \sum_{\mathsf{K} \in \Zstar{26}\times\Z{26}}\Pr[\mathbf{K}_R = \mathsf{K}]\Pr[x=D_\mathsf{K}(y)]&\\
				&=\frac{1}{312}\sum_{\mathsf{K} \in \Zstar{26}\times\Z{26}}\Pr[x=a^{-1}(y-b)\Mod{26}]& \text{(Key is sampled uniformly at random)}&\\
				&=\frac{1}{312}\sum_{\mathsf{K} \in \Zstar{26}\times\Z{26}}\frac{1}{26}&\\
				&=\frac{1}{312}\cdot\frac{312}{26}&\\
				&=\frac{1}{26}&
			\end{flalign*}
			Now, $\forall x,y \in \Z{26}$,
			\begin{flalign*}
				\Pr[\mathbf{C}_R = y | \mathbf{M}_R = x]&= \Pr[\mathbf{K}_R = (a,b): y = ax+b\Mod{26}]&\\
				&=\frac{1}{26}&\text{(For an $a$, there is a fixed $b\in \Z{26}$)}&
			\end{flalign*}
			{\color{blue} Since $\Pr[\mathbf{C}_R = y | \mathbf{M}_R = x] = \Pr[\mathbf{C}_R = y]$, Affine Cipher achieves perfect secrecy when keys are sampled uniformly at random}
			\item[A.4] For any $y \in \{0,1\}^n$,
			\begin{flalign*}
				\Pr[\mathbf{C}_R = y] &= \sum_{\mathsf{K} \in \{0,1\}^n\backslash0^n}\Pr[\mathbf{K}_R = \mathsf{K}]\Pr[x=D_\mathsf{K}(y)]&\\
				&=\frac{1}{2^n - 1}\sum_{\mathsf{K} \in \{0,1\}^n\backslash0^n}\Pr[x=y\oplus \mathsf{K}] &\text{(Key is sampled uniformly at random)}\\
				&=\frac{1}{2^n - 1}\sum_{\mathsf{K} \in \{0,1\}^n\backslash0^n}\frac{1}{2^n}&\\
				&=\frac{1}{2^n - 1}\cdot\frac{1}{2^n}\cdot(2^n-1)&\\
				&=\frac{1}{2^n}
			\end{flalign*}
			Now, $\forall x,y \in \{0,1\}^n$,
			\begin{flalign*}
				\Pr[\mathbf{C}_R = y | \mathbf{M}_R = x]&= \Pr[\mathbf{K}_R = \mathsf{K}: y = x\oplus \mathsf{K}]&\\
				&=\Pr[\mathbf{K}_R = \mathsf{K}: \mathsf{K} = x\oplus y]&\\
				&=\frac{1}{2^n-1}
			\end{flalign*}
			{\color{blue} $\therefore$Since $\Pr[\mathbf{C}_R = y | \mathbf{M}_R = x] \neq \Pr[\mathbf{C}_R = y]$, the scheme is not perfectly secure.}
		\end{description}
	\end{framed}
	
	\item[B. [Entropy : \!\!\!\!  $\mathbf{4 + (4\times 2) + (4\times 2) + (3+1) = 24\ points.}$\!\!\!]]
	
	\begin{framed}
		{\bf Answers}
		\begin{description}
			\item[B.1] We know that,
			\begin{flalign*}
				&H(\mathbf{K}_R, \mathbf{M}_R, \mathbf{C}_R) = H(\mathbf{K}_R, \mathbf{M}_R) = H(\mathbf{K}_R) + H(\mathbf{M}_R) &
				\intertext{And, by theorem}
				&H(\mathbf{K}_R|\mathbf{C}_R) = H(\mathbf{K}_R)+H(\mathbf{M}_R)-H(\mathbf{C}_R)&
			\end{flalign*}
			Using $H(X|Y) = H(X)-H(Y)$,
			\begin{flalign*}
				H(\mathbf{K}_R|\mathbf{C}_R) - H(\mathbf{M}_R|\mathbf{C}_R) &= H(\mathbf{K}_R) + H(\mathbf{M}_R) - H(\mathbf{C}_R) - H(\mathbf{M}_R|\mathbf{C}_R)&\\
				&= H(\mathbf{K}_R, \mathbf{M}_R, \mathbf{C}_R)- H(\mathbf{C}_R) - H(\mathbf{M}_R|\mathbf{C}_R)&\\
				&= H(\mathbf{K}_R, \mathbf{M}_R, \mathbf{C}_R)-H(\mathbf{M}_R,\mathbf{C}_R)&\\
				&= H(\mathbf{K}_R|\mathbf{M}_R,\mathbf{C}_R) \ge 0&
			\end{flalign*}
			{\color{blue}$\therefore H(\mathbf{K}_R|\mathbf{C}_R) \ge H(\mathbf{M}_R|\mathbf{C}_R)$}
			\item[B.2] Since messages and keys are equiprobable, $\Pr(\mathbf{M}_R=\mathsf{M}) = \frac{1}{26}\ \forall \mathsf{M} \in \Z{26}$ and $\Pr(\mathbf{K}_R=\mathsf{K}) = \frac{1}{312}\ \forall \mathsf{K} \in \Zstar{26}\times\Z{26})$. So $\Pr(\mathbf{C}_R = \mathsf{C}) = \frac{1}{26}\ \forall \mathsf{C} \in \Z{26}$.
			\begin{flalign*}
				H(\mathbf{K}_R) &= \sum_{\mathsf{K}\in\Zstar{26}\times\Z{26}}\Pr(\mathbf{K}_R = \mathsf{K})\cdot\log_2\frac{1}{\Pr(\mathbf{K}_R = \mathsf{K})}&\\
				&=\sum_{\mathsf{K}\in\Zstar{26}\times\Z{26}}\frac{1}{312}\cdot\log_2312&\\
				&=\log_2312&\\
				&=8.285\text{ (approx)}&\\
				H(\mathbf{M}_R) &= \log_226&\\
				&=4.7\text{ (approx)}&\\
				H(\mathbf{C}_R) &=4.7\text{ (approx)}&
			\end{flalign*}
			\begin{flalign*}
				H(\mathbf{K}_R|\mathbf{C}_R) &= H(\mathbf{K}_R) + H(\mathbf{M}_R) - H(\mathbf{C}_R)& \text{(Using Theorem)}&\\
				&= \log_2312 = 8.285\text{ (approx)}&
			\end{flalign*}
			{\color{blue}$\therefore H(\mathbf{K}_R|\mathbf{C}_R) = 8.285$}
			\begin{flalign*}
				H(\mathbf{M}_R,\mathbf{C}_R) &=\sum_{\mathsf{C}\in\Z{26}}\sum_{\mathsf{M}\in\Z{26}}Pr(\mathbf{C}_R = \mathsf{C},\mathbf{M}_R = \mathsf{M})\cdot\log_2\frac{1}{Pr(\mathbf{C}_R = \mathsf{C},\mathbf{M}_R = \mathsf{M})} &\\
				&= \sum_{\mathsf{K}\in\Zstar{26}\times\Z{26}}\Pr(\mathbf{K}_R = \mathsf{K})\cdot\log_2\frac{1}{\Pr(\mathbf{K}_R = \mathsf{K})}\text{\qquad\qquad($\mathsf{C}=E_\mathsf{K}(\mathsf{M})$ for some $\mathsf{K}\in\mathbf{K}$)}&\\
				&= H(\mathbf{K}_R) &\\
				&= \log_2312&
			\end{flalign*}
			\begin{flalign*}
				H(\mathbf{K}_R|\mathbf{M}_R,\mathbf{C}_R) &= H(\mathbf{K}_R,\mathbf{M}_R,\mathbf{C}_R)-H(\mathbf{M}_R,\mathbf{C}_R)&\\
				&= H(\mathbf{K}_R)+H(\mathbf{M}_R)-H(\mathbf{K}_R)&\\
				&= \log_226&\\
				&= 4.7 \text{ (approx)}&
			\end{flalign*}
			{\color{blue}$\therefore H(\mathbf{K}_R|\mathbf{M}_R,\mathbf{C}_R) = 4.7$}
			\item[B.3] 
			\begin{itemize}
				\item $f(x) = 7^x$
					\begin{flalign*}
						H(\mathbf{R}) &= \sum_{\mathsf{R}\in R}\Pr(\mathbf{R}=\mathsf{R})\cdot\log_2\frac{1}{\Pr(\mathbf{R}=\mathsf{R})}&\\
						&\ge\sum_{x\in D}\Pr(\mathbf{R}=7^x)\cdot\log_2\frac{1}{\Pr(\mathbf{R}=7^x)}&\text{($f(D)\subseteq R$)}&\\
						&=\sum_{x\in D}\Pr(\mathbf{D}=x)\cdot\log_2\frac{1}{\Pr(\mathbf{D}=x)}&\\
						&=H(\mathbf{D})&
					\end{flalign*}
					{\color{blue}$\therefore H(\mathbf{R}) \ge H(\mathbf{D})$}
			\end{itemize}
			\item[B.4] Let $\underset{z\in Z}{\min} \log_2 \frac{1}{\Pr(\mathbf{Z} = z)} = k$
			\begin{flalign*}
				H(\mathbf{Z}) &= \sum_{z\in Z}\Pr(\mathbf{Z}=z)\cdot\log_2\frac{1}{\Pr(\mathbf{Z}=z)}&\\
				&\ge\sum_{z\in Z}\Pr(\mathbf{Z}=z)\cdot k&\left(\log_2 \frac{1}{\Pr(\mathbf{Z} = z)} \ge \underset{z_1\in Z}{\min} \log_2 \frac{1}{\Pr(\mathbf{Z} = z_1)} \forall z \in Z\right)&\\
				&=k\cdot\sum_{z\in Z}\Pr(\mathbf{Z}=z)&\\
				&=k = \underset{z\in Z}{\min} \log_2 \frac{1}{\Pr(\mathbf{Z} = z)} &\\
				&= -\log_2\left(\underset{z\in Z}{max}\Pr(\mathbf{Z} = z)\right) = H_\infty(\mathbf{Z})&
			\end{flalign*}
			{\color{blue}$\therefore H(\mathbf{Z}) \ge H_\infty(\mathbf{Z}) \ge 0$}\\
			$H(\mathbf{Z}) = H_\infty(\mathbf{Z})$ when $-\log_2\left(\underset{z\in Z}{max}\Pr(\mathbf{Z} = z)\right) = -\log_2\left(\Pr(\mathbf{Z} = z_1)\right) \forall z_1 \in Z$\\
			$\Rightarrow \underset{z\in Z}{max}\Pr(\mathbf{Z} = z) = \Pr(\mathbf{Z} = z_1) \forall z_1 \in Z$\\
			{\color{blue}$\therefore$ If $\mathbf{Z}$ has a uniform distribution, $H(\mathbf{Z}) = H_\infty(\mathbf{Z})$.}
		\end{description}
	\end{framed}
\end{description}
\end{document}
