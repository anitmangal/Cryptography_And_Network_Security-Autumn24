\documentclass{article}
\usepackage{amsmath}				% want AMS fonts
\usepackage{amssymb}                            % use AMS symbols
\usepackage{graphicx}
\usepackage{enumerate,url}
 %\usepackage[ruled]{algorithm2e}
%% preamble.tex
%% this should be included with a command like
%% %% preamble.tex
%% this should be included with a command like
%% %% preamble.tex
%% this should be included with a command like
%% \input{p}

\usepackage{amsfonts}
\usepackage{amsthm}
\usepackage{latexsym}
\usepackage{amsmath}
\usepackage{color}
\usepackage{comment}
\usepackage{enumitem}
\newcommand{\sk}{s}
\newtheorem{theorem}{Theorem}
\newtheorem{lesson}{Lesson}
\newtheorem{proposition}{Proposition}
\newtheorem{lemma}{Lemma}
\newtheorem{corollary}{Corollary}
\newtheorem{fact}{Fact}
\newtheorem*{claim}{Claim}
%\theoremstyle{definition}
\newtheorem{definition}{Definition}
\newtheorem{assumption}{Assumption}
\theoremstyle{remark}
\newtheorem{example}{Example}
\newtheorem*{remark}{Remark}

\DeclareSymbolFont{AMSb}{U}{msb}{m}{n}
\DeclareMathSymbol{\F}{\mathalpha}{AMSb}{"46}
\DeclareMathSymbol{\N}{\mathalpha}{AMSb}{"4E}
\DeclareMathSymbol{\R}{\mathalpha}{AMSb}{"52}
\DeclareMathSymbol{\X}{\mathalpha}{AMSb}{"58}
\DeclareMathSymbol{\Zz}{\mathalpha}{AMSb}{"5A}
\newcommand{\Z}[1]{{\ensuremath{\Zz_{#1}} }}
\newcommand{\Zs}[1]{\ensuremath{\Zz^{\ast}_{#1}}}
\newcommand{\Zn}{\Z{n}}
\newcommand{\Zns}{\Zs{n}}
\newcommand{\Zstar}[1]{\Zs{#1}}
\newcommand{\Zp}{{\Z{p}}}
\newcommand{\Zps}{\Zs{p}}
\newcommand{\Zqs}{\Zs{q}}
\newcommand{\Zq}{{\Z{q}}}
%\newcommand{\QR}{\mathop{\mathrm{QR}}\nolimits}
\newcommand{\ord}[1]{\mathop{\mathrm{ord}}({#1})}
\newcommand{\QR}[1]{\ensuremath{\textit{QR}_{#1}}}
\newcommand{\becomes}{:=}
\newcommand{\rem}[1]{\ensuremath{\ \operatorname{rem} #1}}  
\newcommand{\U}{{\mathcal{U}}}
\newcommand{\floor}[1]{\ensuremath{\lfloor{#1}\rfloor}}
\newcommand{\de}[1]{\ensuremath{\Delta{#1}}}
\newcommand{\js}[2]{\left( \frac{#1}{#2} \right)}

\renewcommand{\QR}{{\mbox{QR}}}
\newcommand{\QNR}{{\mbox{QNR}}}
\newcommand{\crt}{{\mbox{CRT}}}
\newcommand{\rsa}{{\mbox{RSA}}}
\newcommand{\rsamod}{{\mbox{RSA-modulus}}}


\newcommand{\greq}[1]{\stackrel{#1}{=}}
\newcommand{\hash}{\ensuremath{\mathcal{H}}}
\newcommand{\negl}{{\tt neg}}
\newcommand{\cindist}{\stackrel{c}{\approx}}
\newcommand{\A}{{\mathcal{A}}}
\newcommand{\B}{{\mathcal{B}}}

\newcommand{\gen}{\mathsf{Gen}}
\newcommand{\keygen}{\mathsf{KeyGen}}
\newcommand{\RSA}{\mathsf{RSA}}
\newcommand{\pk}{\mathsf{pk}}
\newcommand{\PK}{\mathsf{PK}}
\newcommand{\SK}{\mathsf{SK}}
\newcommand{\CT}{\mathsf{CT}}
\newcommand{\sign}{\mathsf{Sign}}
\newcommand{\verify}{\mathsf{Verify}}
\newcommand{\enc}{\mathsf{Enc}}
\newcommand{\dec}{\mathsf{Dec}}
\newcommand{\mac}{\mathsf{Tag}}

\newcommand{\ASETUP}{\textsf{ABE.Setup}}
\newcommand{\AEXTRACT}{\textsf{ABE.Extract}}
\newcommand{\AENCRYPT}{\textsf{ABE.Enc}}
\newcommand{\ADECRYPT}{\textsf{ABE.Dec}}
\newcommand{\construction}{\mathit{Construction}}
\newcommand{\domain}{\mathit{Domain}}

\def\matA{\mathbf{A}}
\def\matT{\mathbf{T}}
\def\matB{\mathbf{B}}
\def\matF{\mathbf{F}}
\def\matH{\mathbf{H}}
\def\matM{\mathbf{M}}
\def\matS{\mathbf{S}}

\def\veca{\mathbf{a}}
\def\vecb{\mathbf{b}}
\def\vecc{\mathbf{c}}
\def\vecd{\mathbf{d}}
\def\vece{\mathbf{e}}
\def\vecu{\mathbf{u}}
\def\vecr{\mathbf{r}}
\def\vecm{{m}}
\def\vecs{\mathbf{s}}
\def\vect{{t}}
\def\vecv{{v}}
\def\vecw{\mathbf{w}}
\def\vecx{\mathbf{x}}
\def\vecy{\mathbf{y}}
\def\vecgamma{\mathbf{\gamma}}
\def\halfq{\left\lfloor\frac{q}{2}\right\rfloor}

\newcommand{\ZQ}{\mathbb{Z}_q}
\newcommand{\G}{\mathbb{G}}
\def\Q{\mathbb{Q}}
\newcommand{\nat}{\mathbb{N}}
\newcommand{\intg}{\mathbb{Z}}
\newcommand{\detm}{\mathsf{det}}

\newcommand{\getsr}{\overset{R}{\leftarrow}}

\setlength{\oddsidemargin}{.25in}
\setlength{\evensidemargin}{.25in}
\setlength{\textwidth}{6in}
\setlength{\topmargin}{-0.4in}
\setlength{\textheight}{8.5in}

\newcommand{\handout}[5]{
  % \renewcommand{\thepage}{#1-\arabic{page}}
   \noindent
   \begin{center}
   \framebox{
      \vbox{
    \hbox to 5.78in { {\bf CS60065: Cryptography and Network Security} \hfill {\it #2 }}
       \vspace{4mm}
       \hbox to 5.78in { {\Large \hfill #5  \hfill} }
       \vspace{2mm}
       \hbox to 5.78in { {\it #3 \hfill #4} }
      }
   }
   \end{center}
   \vspace*{4mm}
}

\newcommand{\ho}[4]{\handout{#1}{#2}{{\sf Instructor:}
#3}{}{Handout #1: #4}}

\newcommand{\lnotes}[3]{\handout{#1}{#2}{{\sf Instructor:}
#3}{}{Lecture #1}}


\newcommand{\quiz}[3]{\handout{#1}{#2}{{\sf Instructor:} #3}{Total: 40 pts}{Mid-Term #1}}

\newcommand{\major}[3]{\handout{#1}{#2}{{\sf Instructor:}
#3}{}{Major #1}}

\newcommand{\solution}[1]{#1}
%\homework{number}{out}{due}{instructor}
\newcommand{\homework}[4]{\handout{HW #1}{#2}{{\sf Instructor:} #4}{{\sf Due:} #3}{\textbf{Assignment #1}}}
\newcommand{\homeworksol}[4]{\handout{HW #1}{#2}{Name: #4}{{\sf Due:} #3}{Assignment #1}}

%================================================
% problemset macros
%================================================
% count problems
\newcounter{solutioncount}
\setcounter{solutioncount}{0}
\newcommand{\problem}[1]{%
\addtocounter{solutioncount}{1}%
\section*{Problem \arabic{solutioncount}: #1}}

% lets you make alphabetical lists (at first-level of enumeration)
\newenvironment
  {alphabetize}{\renewcommand{\theenumi}{\alph{enumi}}\begin{enumerate}}
  {\end{enumerate}\renewcommand{\theenumi}{\arabic{enumi}}}


\usepackage{amsfonts}
\usepackage{amsthm}
\usepackage{latexsym}
\usepackage{amsmath}
\usepackage{color}
\usepackage{comment}
\usepackage{enumitem}
\newcommand{\sk}{s}
\newtheorem{theorem}{Theorem}
\newtheorem{lesson}{Lesson}
\newtheorem{proposition}{Proposition}
\newtheorem{lemma}{Lemma}
\newtheorem{corollary}{Corollary}
\newtheorem{fact}{Fact}
\newtheorem*{claim}{Claim}
%\theoremstyle{definition}
\newtheorem{definition}{Definition}
\newtheorem{assumption}{Assumption}
\theoremstyle{remark}
\newtheorem{example}{Example}
\newtheorem*{remark}{Remark}

\DeclareSymbolFont{AMSb}{U}{msb}{m}{n}
\DeclareMathSymbol{\F}{\mathalpha}{AMSb}{"46}
\DeclareMathSymbol{\N}{\mathalpha}{AMSb}{"4E}
\DeclareMathSymbol{\R}{\mathalpha}{AMSb}{"52}
\DeclareMathSymbol{\X}{\mathalpha}{AMSb}{"58}
\DeclareMathSymbol{\Zz}{\mathalpha}{AMSb}{"5A}
\newcommand{\Z}[1]{{\ensuremath{\Zz_{#1}} }}
\newcommand{\Zs}[1]{\ensuremath{\Zz^{\ast}_{#1}}}
\newcommand{\Zn}{\Z{n}}
\newcommand{\Zns}{\Zs{n}}
\newcommand{\Zstar}[1]{\Zs{#1}}
\newcommand{\Zp}{{\Z{p}}}
\newcommand{\Zps}{\Zs{p}}
\newcommand{\Zqs}{\Zs{q}}
\newcommand{\Zq}{{\Z{q}}}
%\newcommand{\QR}{\mathop{\mathrm{QR}}\nolimits}
\newcommand{\ord}[1]{\mathop{\mathrm{ord}}({#1})}
\newcommand{\QR}[1]{\ensuremath{\textit{QR}_{#1}}}
\newcommand{\becomes}{:=}
\newcommand{\rem}[1]{\ensuremath{\ \operatorname{rem} #1}}  
\newcommand{\U}{{\mathcal{U}}}
\newcommand{\floor}[1]{\ensuremath{\lfloor{#1}\rfloor}}
\newcommand{\de}[1]{\ensuremath{\Delta{#1}}}
\newcommand{\js}[2]{\left( \frac{#1}{#2} \right)}

\renewcommand{\QR}{{\mbox{QR}}}
\newcommand{\QNR}{{\mbox{QNR}}}
\newcommand{\crt}{{\mbox{CRT}}}
\newcommand{\rsa}{{\mbox{RSA}}}
\newcommand{\rsamod}{{\mbox{RSA-modulus}}}


\newcommand{\greq}[1]{\stackrel{#1}{=}}
\newcommand{\hash}{\ensuremath{\mathcal{H}}}
\newcommand{\negl}{{\tt neg}}
\newcommand{\cindist}{\stackrel{c}{\approx}}
\newcommand{\A}{{\mathcal{A}}}
\newcommand{\B}{{\mathcal{B}}}

\newcommand{\gen}{\mathsf{Gen}}
\newcommand{\keygen}{\mathsf{KeyGen}}
\newcommand{\RSA}{\mathsf{RSA}}
\newcommand{\pk}{\mathsf{pk}}
\newcommand{\PK}{\mathsf{PK}}
\newcommand{\SK}{\mathsf{SK}}
\newcommand{\CT}{\mathsf{CT}}
\newcommand{\sign}{\mathsf{Sign}}
\newcommand{\verify}{\mathsf{Verify}}
\newcommand{\enc}{\mathsf{Enc}}
\newcommand{\dec}{\mathsf{Dec}}
\newcommand{\mac}{\mathsf{Tag}}

\newcommand{\ASETUP}{\textsf{ABE.Setup}}
\newcommand{\AEXTRACT}{\textsf{ABE.Extract}}
\newcommand{\AENCRYPT}{\textsf{ABE.Enc}}
\newcommand{\ADECRYPT}{\textsf{ABE.Dec}}
\newcommand{\construction}{\mathit{Construction}}
\newcommand{\domain}{\mathit{Domain}}

\def\matA{\mathbf{A}}
\def\matT{\mathbf{T}}
\def\matB{\mathbf{B}}
\def\matF{\mathbf{F}}
\def\matH{\mathbf{H}}
\def\matM{\mathbf{M}}
\def\matS{\mathbf{S}}

\def\veca{\mathbf{a}}
\def\vecb{\mathbf{b}}
\def\vecc{\mathbf{c}}
\def\vecd{\mathbf{d}}
\def\vece{\mathbf{e}}
\def\vecu{\mathbf{u}}
\def\vecr{\mathbf{r}}
\def\vecm{{m}}
\def\vecs{\mathbf{s}}
\def\vect{{t}}
\def\vecv{{v}}
\def\vecw{\mathbf{w}}
\def\vecx{\mathbf{x}}
\def\vecy{\mathbf{y}}
\def\vecgamma{\mathbf{\gamma}}
\def\halfq{\left\lfloor\frac{q}{2}\right\rfloor}

\newcommand{\ZQ}{\mathbb{Z}_q}
\newcommand{\G}{\mathbb{G}}
\def\Q{\mathbb{Q}}
\newcommand{\nat}{\mathbb{N}}
\newcommand{\intg}{\mathbb{Z}}
\newcommand{\detm}{\mathsf{det}}

\newcommand{\getsr}{\overset{R}{\leftarrow}}

\setlength{\oddsidemargin}{.25in}
\setlength{\evensidemargin}{.25in}
\setlength{\textwidth}{6in}
\setlength{\topmargin}{-0.4in}
\setlength{\textheight}{8.5in}

\newcommand{\handout}[5]{
  % \renewcommand{\thepage}{#1-\arabic{page}}
   \noindent
   \begin{center}
   \framebox{
      \vbox{
    \hbox to 5.78in { {\bf CS60065: Cryptography and Network Security} \hfill {\it #2 }}
       \vspace{4mm}
       \hbox to 5.78in { {\Large \hfill #5  \hfill} }
       \vspace{2mm}
       \hbox to 5.78in { {\it #3 \hfill #4} }
      }
   }
   \end{center}
   \vspace*{4mm}
}

\newcommand{\ho}[4]{\handout{#1}{#2}{{\sf Instructor:}
#3}{}{Handout #1: #4}}

\newcommand{\lnotes}[3]{\handout{#1}{#2}{{\sf Instructor:}
#3}{}{Lecture #1}}


\newcommand{\quiz}[3]{\handout{#1}{#2}{{\sf Instructor:} #3}{Total: 40 pts}{Mid-Term #1}}

\newcommand{\major}[3]{\handout{#1}{#2}{{\sf Instructor:}
#3}{}{Major #1}}

\newcommand{\solution}[1]{#1}
%\homework{number}{out}{due}{instructor}
\newcommand{\homework}[4]{\handout{HW #1}{#2}{{\sf Instructor:} #4}{{\sf Due:} #3}{\textbf{Assignment #1}}}
\newcommand{\homeworksol}[4]{\handout{HW #1}{#2}{Name: #4}{{\sf Due:} #3}{Assignment #1}}

%================================================
% problemset macros
%================================================
% count problems
\newcounter{solutioncount}
\setcounter{solutioncount}{0}
\newcommand{\problem}[1]{%
\addtocounter{solutioncount}{1}%
\section*{Problem \arabic{solutioncount}: #1}}

% lets you make alphabetical lists (at first-level of enumeration)
\newenvironment
  {alphabetize}{\renewcommand{\theenumi}{\alph{enumi}}\begin{enumerate}}
  {\end{enumerate}\renewcommand{\theenumi}{\arabic{enumi}}}


\usepackage{amsfonts}
\usepackage{amsthm}
\usepackage{latexsym}
\usepackage{amsmath}
\usepackage{color}
\usepackage{comment}
\usepackage{enumitem}
\newcommand{\sk}{s}
\newtheorem{theorem}{Theorem}
\newtheorem{lesson}{Lesson}
\newtheorem{proposition}{Proposition}
\newtheorem{lemma}{Lemma}
\newtheorem{corollary}{Corollary}
\newtheorem{fact}{Fact}
\newtheorem*{claim}{Claim}
%\theoremstyle{definition}
\newtheorem{definition}{Definition}
\newtheorem{assumption}{Assumption}
\theoremstyle{remark}
\newtheorem{example}{Example}
\newtheorem*{remark}{Remark}

\DeclareSymbolFont{AMSb}{U}{msb}{m}{n}
\DeclareMathSymbol{\F}{\mathalpha}{AMSb}{"46}
\DeclareMathSymbol{\N}{\mathalpha}{AMSb}{"4E}
\DeclareMathSymbol{\R}{\mathalpha}{AMSb}{"52}
\DeclareMathSymbol{\X}{\mathalpha}{AMSb}{"58}
\DeclareMathSymbol{\Zz}{\mathalpha}{AMSb}{"5A}
\newcommand{\Z}[1]{{\ensuremath{\Zz_{#1}} }}
\newcommand{\Zs}[1]{\ensuremath{\Zz^{\ast}_{#1}}}
\newcommand{\Zn}{\Z{n}}
\newcommand{\Zns}{\Zs{n}}
\newcommand{\Zstar}[1]{\Zs{#1}}
\newcommand{\Zp}{{\Z{p}}}
\newcommand{\Zps}{\Zs{p}}
\newcommand{\Zqs}{\Zs{q}}
\newcommand{\Zq}{{\Z{q}}}
%\newcommand{\QR}{\mathop{\mathrm{QR}}\nolimits}
\newcommand{\ord}[1]{\mathop{\mathrm{ord}}({#1})}
\newcommand{\QR}[1]{\ensuremath{\textit{QR}_{#1}}}
\newcommand{\becomes}{:=}
\newcommand{\rem}[1]{\ensuremath{\ \operatorname{rem} #1}}  
\newcommand{\U}{{\mathcal{U}}}
\newcommand{\floor}[1]{\ensuremath{\lfloor{#1}\rfloor}}
\newcommand{\de}[1]{\ensuremath{\Delta{#1}}}
\newcommand{\js}[2]{\left( \frac{#1}{#2} \right)}

\renewcommand{\QR}{{\mbox{QR}}}
\newcommand{\QNR}{{\mbox{QNR}}}
\newcommand{\crt}{{\mbox{CRT}}}
\newcommand{\rsa}{{\mbox{RSA}}}
\newcommand{\rsamod}{{\mbox{RSA-modulus}}}


\newcommand{\greq}[1]{\stackrel{#1}{=}}
\newcommand{\hash}{\ensuremath{\mathcal{H}}}
\newcommand{\negl}{{\tt neg}}
\newcommand{\cindist}{\stackrel{c}{\approx}}
\newcommand{\A}{{\mathcal{A}}}
\newcommand{\B}{{\mathcal{B}}}

\newcommand{\gen}{\mathsf{Gen}}
\newcommand{\keygen}{\mathsf{KeyGen}}
\newcommand{\RSA}{\mathsf{RSA}}
\newcommand{\pk}{\mathsf{pk}}
\newcommand{\PK}{\mathsf{PK}}
\newcommand{\SK}{\mathsf{SK}}
\newcommand{\CT}{\mathsf{CT}}
\newcommand{\sign}{\mathsf{Sign}}
\newcommand{\verify}{\mathsf{Verify}}
\newcommand{\enc}{\mathsf{Enc}}
\newcommand{\dec}{\mathsf{Dec}}
\newcommand{\mac}{\mathsf{Tag}}

\newcommand{\ASETUP}{\textsf{ABE.Setup}}
\newcommand{\AEXTRACT}{\textsf{ABE.Extract}}
\newcommand{\AENCRYPT}{\textsf{ABE.Enc}}
\newcommand{\ADECRYPT}{\textsf{ABE.Dec}}
\newcommand{\construction}{\mathit{Construction}}
\newcommand{\domain}{\mathit{Domain}}

\def\matA{\mathbf{A}}
\def\matT{\mathbf{T}}
\def\matB{\mathbf{B}}
\def\matF{\mathbf{F}}
\def\matH{\mathbf{H}}
\def\matM{\mathbf{M}}
\def\matS{\mathbf{S}}

\def\veca{\mathbf{a}}
\def\vecb{\mathbf{b}}
\def\vecc{\mathbf{c}}
\def\vecd{\mathbf{d}}
\def\vece{\mathbf{e}}
\def\vecu{\mathbf{u}}
\def\vecr{\mathbf{r}}
\def\vecm{{m}}
\def\vecs{\mathbf{s}}
\def\vect{{t}}
\def\vecv{{v}}
\def\vecw{\mathbf{w}}
\def\vecx{\mathbf{x}}
\def\vecy{\mathbf{y}}
\def\vecgamma{\mathbf{\gamma}}
\def\halfq{\left\lfloor\frac{q}{2}\right\rfloor}

\newcommand{\ZQ}{\mathbb{Z}_q}
\newcommand{\G}{\mathbb{G}}
\def\Q{\mathbb{Q}}
\newcommand{\nat}{\mathbb{N}}
\newcommand{\intg}{\mathbb{Z}}
\newcommand{\detm}{\mathsf{det}}

\newcommand{\getsr}{\overset{R}{\leftarrow}}

\setlength{\oddsidemargin}{.25in}
\setlength{\evensidemargin}{.25in}
\setlength{\textwidth}{6in}
\setlength{\topmargin}{-0.4in}
\setlength{\textheight}{8.5in}

\newcommand{\handout}[5]{
  % \renewcommand{\thepage}{#1-\arabic{page}}
   \noindent
   \begin{center}
   \framebox{
      \vbox{
    \hbox to 5.78in { {\bf CS60065: Cryptography and Network Security} \hfill {\it #2 }}
       \vspace{4mm}
       \hbox to 5.78in { {\Large \hfill #5  \hfill} }
       \vspace{2mm}
       \hbox to 5.78in { {\it #3 \hfill #4} }
      }
   }
   \end{center}
   \vspace*{4mm}
}

\newcommand{\ho}[4]{\handout{#1}{#2}{{\sf Instructor:}
#3}{}{Handout #1: #4}}

\newcommand{\lnotes}[3]{\handout{#1}{#2}{{\sf Instructor:}
#3}{}{Lecture #1}}


\newcommand{\quiz}[3]{\handout{#1}{#2}{{\sf Instructor:} #3}{Total: 40 pts}{Mid-Term #1}}

\newcommand{\major}[3]{\handout{#1}{#2}{{\sf Instructor:}
#3}{}{Major #1}}

\newcommand{\solution}[1]{#1}
%\homework{number}{out}{due}{instructor}
\newcommand{\homework}[4]{\handout{HW #1}{#2}{{\sf Instructor:} #4}{{\sf Due:} #3}{\textbf{Assignment #1}}}
\newcommand{\homeworksol}[4]{\handout{HW #1}{#2}{Name: #4}{{\sf Due:} #3}{Assignment #1}}

%================================================
% problemset macros
%================================================
% count problems
\newcounter{solutioncount}
\setcounter{solutioncount}{0}
\newcommand{\problem}[1]{%
\addtocounter{solutioncount}{1}%
\section*{Problem \arabic{solutioncount}: #1}}

% lets you make alphabetical lists (at first-level of enumeration)
\newenvironment
  {alphabetize}{\renewcommand{\theenumi}{\alph{enumi}}\begin{enumerate}}
  {\end{enumerate}\renewcommand{\theenumi}{\arabic{enumi}}}

\usepackage{framed} %custom addition
\newcommand{\Mod}[1]{\ (\mathrm{mod}\ #1)}

\raggedbottom

\begin{document}
\raggedbottom
\homeworksol{1}{{\sf Total:} $70$ points}{Aug 12, 2024}{Anit Mangal}
\vspace*{-0.4cm}
{ \sf \textbf{{\color{red}Note}}: {\color{blue} The basic policies are stated in the course page. Using GPT (or similar tools) to solve problems from the assignment is \textbf{strictly prohibited}. Use of any other (possibly online) source(s) \textbf{must} be clearly stated in the solution. \textbf{Any dishonesty, if caught, will yield zero credits for the entire assignment.}}}\\\vspace*{-0.2cm}
\hrule
\begin{description}[leftmargin=*]
	\item[A. [Modular Arithmetic : \!\!\!\!  $\mathbf{3 \times 3 = 9\ points.}$\!\!\!]] 
	%\begin{comment}	%solution space
	\begin{framed}
		{\bf Answers}
		\begin{description}
			\item[A.1] 
			\begin{flalign*}
					25 &\equiv 1 \Mod{8} &\\
				\intertext{We know that, if $a \equiv b\Mod{m}\Rightarrow ac \equiv bc \Mod{m}$. \newline Multiplying by $25$}
					\Rightarrow25^2 &\equiv 25 \Mod{8} \\
						&\equiv 1 \Mod{8} \\
					25^3 &\equiv 25 \Mod{8} \\
						&\equiv 1 \Mod{8} \\
						&\vdots\\
					25^n &\equiv 1 \Mod{8} \\
					5^{2n} &\equiv 1 \Mod{8}
				\intertext{Using $a \equiv b \Mod{m} \Rightarrow a+c \equiv b+c\Mod{m}$}
					\Rightarrow5^{2n} + 7 &\equiv (1+7) \Mod{8}\\
						&\equiv 0 \Mod{8}
			\end{flalign*}
			{\color{blue}$\therefore 8 \mid (5^{2n}+7), \forall n \in \N.$}
			\newpage
			\item[A.2]
			\begin{flalign}
					7 &\equiv 3 \Mod{4}\nonumber &\\
				\intertext{We know that, if $a \equiv b\Mod{m}\Rightarrow ac \equiv bc \Mod{m}$. \newline Multiplying by $7$}
					\Rightarrow 7^2 &\equiv 3\cdot7 \Mod{4} \nonumber\\
						&\equiv 21 \Mod{4} \nonumber\\
						&\equiv 1 \Mod{4} \nonumber\\
					7^4 &\equiv 1\cdot7^2 \Mod{4} \nonumber\\
						&\equiv 49 \Mod{4} \nonumber\\
						&\equiv 1 \Mod{4} \nonumber\\
					7^6 &\equiv 1\cdot7^2 \Mod{4} \nonumber\\
						&\equiv 49 \Mod{4} \nonumber\\
						&\equiv 1 \Mod{4} \nonumber\\
						&\vdots\nonumber\\
					7^{32} &\equiv 1 \Mod{4} \nonumber\\
					\Rightarrow6\cdot7^{32} &\equiv 6 \Mod{4} \equiv 2 \Mod{4}
				\intertext{Now,}
					9 &\equiv 1 \Mod{4} \nonumber\\
					\Rightarrow9^2 &\equiv 1\cdot9 \Mod{4} \nonumber\\
						&\equiv 9 \Mod{4} \nonumber\\
						&\equiv 1 \Mod{4} \nonumber\\
					9^4 &\equiv 1\cdot81 \Mod{4} \nonumber\\
						&\equiv 81 \Mod{4} \nonumber\\
						&\equiv 1 \Mod{4} \nonumber\\
						&\vdots\nonumber\\
					9^{44} &\equiv 1 \Mod{4} \nonumber\\
					9^{45} &\equiv 1\cdot9 \Mod{4} \nonumber\\
						&\equiv 9 \Mod{4} \nonumber\\
						&\equiv 1 \Mod{4} \nonumber\\
					\Rightarrow7\cdot9^{45} &\equiv 7 \Mod{4} \equiv 3 \Mod{4}
			\end{flalign}
			Using $a \equiv b \Mod{m}$ and $c \equiv d \Mod{m}\Rightarrow a+c \equiv b+d \Mod{m}$\newline Adding $(1)$ and $(2)$
			\begin{flalign*}
				\Rightarrow(6\cdot7^{32}+7\cdot9^{45}) &\equiv(2+3)\Mod{4} &\\
					&\equiv5\Mod{4} \\
					&\equiv1\Mod{4}
			\end{flalign*}
			{\color{blue}$\therefore$ remainder when $6\cdot7^{32}+7\cdot9^{45}$ is divided by $4$ is $1$}.
			\clearpage
			\item[A.3]
			\begin{flalign*}
				4n^2 + 1 &\equiv 0 \Mod{65} &\\
				4n^2 &\equiv -1 \Mod{65} \\
				4n^2 &\equiv 64 \Mod{65} \\
				n^2 &\equiv 16 \Mod{65}
				\intertext{One solution for this is}
				n &\equiv \pm4 \Mod{65}
			\end{flalign*}
			{\color{blue}$\Rightarrow n = 65k+4 \text{ or } 65k+61 , \forall k \in \N_0$. \\ Hence, there exist infinitely many $n \in \N$ such that $4n^2+1$ is divisible by $65$.}
			\clearpage
		\end{description}
	\end{framed}
	%\end{comment}
	\clearpage
	\item[B. [GCD and Euclid's Algorithm : \!\!\!\!  $\mathbf{5 + 2 + (3\times 3) = 16 \ points.}$\!\!\!]] 

	%\begin{comment}
	\begin{framed}
		{\bf Answers}
		\begin{description}
			\item[B.1] 
			$ a = -85652,\ b = 16261$ \\
			Using {\it Euclidean division algorithm}, computing $GCD(a,b)$ \\
			Since $a < b$, swapping $a$ and $b$
			\begin{flalign*}
				16261 &= -85652 \times 0 \ + 16261 &\\
				-85652 &= 16261 \times -6 \ + 11914 \\
				16261 &= 11914 \times 1 \ + 4347 \\
				11914 &= 4347 \times 2 \ + 3220 \\
				4347 &= 3220 \times 1 \ + 1127 \\
				3220 &= 1127 \times 2 \ + 966 \\
				1127 &= 966 \times 1 \ + 161 \\
				966 &= 161 \times 6 \ + 0
			\end{flalign*}
			{\color{blue} So, $GCD(a,b) = 161$.}\\ \\
			Using {\it Extended Euclidean Algorithm},
			\begin{equation*}
			\text{\quad}GCD(a,b) = GCD(b, a\%b),\ a \ge b
			\end{equation*}
			And, by {\it B\'ezout's Identity},
			\begin{equation*}
			\text{\quad}\exists x,y \in \Z{} \text{ such that } GCD(a,b) = ax+by
			\end{equation*}
			Let $x_1,y_1$ be B\'ezout's coefficients for $GCD(b, a\%b)$
			\begin{flalign*}
				\therefore ax+by &= bx_1 + (a-\left\lfloor{\frac{a}{b}}\right\rfloor b)y_1 &\\
				ax+by &= ay_1 + b(x_1 - \left\lfloor{\frac{a}{b}}\right\rfloor y_1)
			\end{flalign*}
			\begin{flalign}
				\boxed{
				\Rightarrow x = y_1, y = x_1 - \left\lfloor{\frac{a}{b}}\right\rfloor y_1} &&
			\end{flalign}
			Using this result recursively, B\'ezout's coefficients for $a,b$ can be found. \\
			For $ a = -85652,\ b = 16261$,
			\begin{flalign*}
				&GCD(-85652, 16261) &\longrightarrow x_1,y_1 &&\\
				&= GCD(16261, 11914) &\longrightarrow x_2,y_2 &&\\
				&= GCD(11914, 4347) &\longrightarrow x_3,y_3 &&\\
				&= GCD(4347, 3220) &\longrightarrow x_4,y_4 &&\\
				&= GCD(3220, 1127) &\longrightarrow x_5,y_5 &&\\
				&= GCD(1127, 966) &\longrightarrow x_6,y_6 &&\\
				&= GCD(966, 161) &\longrightarrow x_7,y_7 &&\\
				&= 161&
			\end{flalign*}
			Since $966\cdot(1) + 161\cdot(-5) = 161$
			\begin{flalign*}
				x_7 = 1 &, y_7 = -5
				\intertext{Using result $(3)$ recursively,}
				x_6 = -5 &, y_6 = 1 - \left\lfloor{\frac{1127}{966}}\right\rfloor\cdot(-5) = 6 \\
				x_5 = 6 &, y_5 = -5 - \left\lfloor{\frac{3220}{1127}}\right\rfloor\cdot(6) = -17 \\
				x_4 = -17 &, y_4 = 6 - \left\lfloor{\frac{4347}{3220}}\right\rfloor\cdot(-17) = 23 \\
				x_3 = 23 &, y_3 = -17 - \left\lfloor{\frac{11914}{4347}}\right\rfloor\cdot(23) = -63 \\
				x_2 = -63 &, y_2 = 23 - \left\lfloor{\frac{16261}{11914}}\right\rfloor\cdot(-63) = 86 \\
				x_1 = 86 &, y_1 = -63 - \left\lfloor{\frac{-85652}{16261}}\right\rfloor\cdot(86) = 453
			\end{flalign*}
			$\therefore GCD(a,b)$ expressed as linear combination of $a$ and $b$ is
			{\color{blue}
				\begin{flalign}
					-85652\cdot86 + 16261\cdot453 = 161
				\end{flalign}
			}
			Now, $LCM(-85652, 16261) = -8650852$, \\
			Adding and subtracting $-8650852k$ in (4), $\forall k \in \Z{}$
			\begin{flalign*}
				&(-85652)\cdot86 + 16261\cdot453 + (-8650852k) - (-8650852k) = 161 &\\
				&(-85652)\cdot86 + 16261\cdot453 + (-85652)\cdot(101k) - 16261\cdot(-532k) = 161 &\\
				&(-85652)\cdot(86+101k) + 16261\cdot(453+532k) = 161 &\\
				\Rightarrow & x = 86+101k, y = 453+532k, \forall k \in \Z{} &
			\end{flalign*}
			{\color{blue}$\therefore \ S = \{(86+101k, 453+532k) \mid k \in \Z{}\}$}\\
			Let $GCD(a,b) = d$ and $p \in \mathbb{P}$ be the smallest prime number greater than d \\
			Since p is prime, $GCD(d, p) = 1$\\
			Using B\'ezout's Identity,
			\begin{flalign*}
				&\exists x,y \in \Z{} \text{ such that } dx + py = 1&\\
				\Rightarrow & dx \equiv 1 \Mod{p} &\\
				\Rightarrow & x \equiv (d^{-1}) \Mod{p}
			\end{flalign*}
			So, x (B\'ezout's coefficient of d) is $d^{-1}$. \\
			For $ a = -85652,\ b = 16261$,
			\begin{flalign*}
				& d = 161 \Rightarrow p = 163&\\
				& GCD(163, 161) &&\longrightarrow x,y&&\\
				& = GCD(161, 2) &&\longrightarrow x_1,y_1&&\\
				& = 1&\\
				& \text{Now, }161\cdot(1) + 2\cdot(-80) = 1 &\\
				&\Rightarrow x_1 = 1, y_1 = -80 &\\
				& x = -80, y = 1 - \left\lfloor{\frac{163}{161}}\right\rfloor\cdot(-80) = 81 &\\
				& 163\cdot(-80) + 161\cdot(81) = 1 &
			\end{flalign*}
			{\color{blue} $\therefore (161)^{-1} \Mod{163} = 81$}
			\newpage
			\item[B.2] $gcd^{+}: \N\times\N\rightarrow\N $
			\begin{flalign*}
				&\text{Checking if } gcd^{+} \text{is injective.} &\\
				\intertext{It is sufficient to show that if $\forall (a_1,b_1),(a_2,b_2) \in \N^2$,\newline $gcd^{+}(a_1,b_1) = gcd^{+}(a_2,b_2) \Rightarrow a_1 = a_2 \text{ and } b_1 = b_2$}
				&\text{Take } a_1 = 2, b_1 = 4 &\\
				&\text{\qquad}\Rightarrow gcd^{+}(a_1,b_1) = 2 &\\
				&\text{Take } a_2 = 2, b_2 = 6 &\\
				&\text{\qquad}\Rightarrow gcd^{+}(a_2,b_2) = 2 &\\
				&\text{So, } \exists (a_1,b_1),(a_2,b_2) \in \N^2 \text{ such that } gcd^{+}(a_1,b_1) = gcd^{+}(a_2,b_2) \text{ and } (a_1,b_1) \ne (a_2,b_2). &\\
				&\text{\color{blue}$\therefore gcd^{+}$ is not injective.}&\\ \\
				\intertext{Checking if $gcd^{+}$ is surjective.}
				&\text{Take any $d \in \N$.} &\\
				&\text{We need to prove that $\exists (a,b) \in \N^2$ such that $gcd^{+}(a,b) = d$.} &\\
				&\text{Let } a=d, b=2d&\\
				&\text{\qquad}\Rightarrow gcd^{+}(a,b) = gcd^{+}(d, 2d) = d&\\
				&\text{So, } \forall d \in \N, \exists (a,b) \in \N^2: gcd^{+}(a,b) = d.&
			\end{flalign*}
			{\color{blue} $\therefore gcd^{+}$ is surjective.}
			\clearpage
			\item[B.3]
			\begin{enumerate}
				\item[a]
				\begin{flalign*}
					&gcd(a,b) = d&\\
					&\text{So } a=dk_1,\ b=dk_2;\ gcd(k_1,k_2) = 1 \text{( by definition of gcd)} &\\
					&\text{\qquad}\Rightarrow a^2=d^2k_1^2, b^2=d^2k_2^2&
					\intertext{Since $k_1$ and $k_2$ are co-prime ($gcd(k_1, k_2) = 1$),}
					&gcd(k_1^2, k_2^2) = 1 \text{ (No common factors)}&\\
					&\Rightarrow gcd(a^2,b^2) = gcd(d^2k_1^2, d^2k_2^2) = d^2gcd(k_1^2,k_2^2) = d^2 &\\
					&{\color{blue}\therefore gcd(a^2, b^2) = d^2.}&
				\end{flalign*}
				\item[b]
				\begin{flalign*}
					&\text{Let }gcd(a+b, a-b) = c &\\
					&\text{\qquad}\Rightarrow a+b = cd,\ a-b = ce;\ gcd(d,e) = 1&
				\end{flalign*}
				\begin{tabular}{c c}
					$cd + ce = (a+b) + (a-b)$ & $cd-ce = (a+b) - (a-b)$\\
					$c(d + e) = 2a$ & $c(d-e) = 2b$
				\end{tabular}
				\begin{flalign*}
					&\Rightarrow c \text{ is a factor of } 2a \text{ and } 2b &\\
					& \text{Since } gcd(a,b) = 1 \Rightarrow \ gcd(2a,2b) = 2 &\\
					& \Rightarrow \text{c is a factor of 2} &\\
					& \Rightarrow c = 1 \text{ or } 2 & 
				\end{flalign*}
				{\color{blue}$\therefore gcd(a+b, a-b) = 1 \text{ or } 2$}
				\item[c]
				\begin{flalign*}
					&\text{Let }gcd(3a+2b, 2a+5b) = c &\\
					&\text{\qquad}\Rightarrow 3a+2b = cd,\ 2a+5b = ce;\ gcd(d,e) = 1&\\
					&c(5d-2e) = 11a \text{ and } c(3e-2d) = 11b&\\
					&\Rightarrow c \text{ is a factor of } 11a \text{ and } 11b &\\
					& \text{Since } gcd(a,b) = 1 \Rightarrow \ gcd(11a,11b) = 11 &\\
					& \Rightarrow c\text{ is a factor of 11} &\\
					& \Rightarrow c = 1 \text{ or } 11 & 
				\end{flalign*}
				{\color{blue}$\therefore gcd(3a+2b, 2a+5b) = 1 \text{ or } 11$}
			\end{enumerate}
		\end{description}
	\end{framed}
	%\end{comment}
	\clearpage
	\item[C. [Algebraic Structures : \!\!\!\!  $\mathbf{3 \times 5 = 15\ points.}$\!\!\!]]
	
	%\begin{comment}
	\begin{framed}
		{\bf Answers}
		\begin{description}
			\item[C.1] 
			\begin{flalign*}
				\text{Number of binary operations} &= \prod_{(a,b)\in S}(\text{Number of ways to choose $c\in S$ such that $a\star b = c$}) &\\
				&= \prod_{(a,b)\in S}(n) &\\
				&= {\color{blue}n^{n^2}} &
			\end{flalign*}
			\begin{flalign*}
			\text{Number of commutative operations} &= \prod_{(a,a)\in S}(n)\cdot\prod_{(a,b)\in S, a<b}(n) \text{\quad[Value for $b\star a = a\star b$]}&\\
			&= (n)^n \cdot (n)^{\frac{n^2 - n}{2}} &\\
			&= {\color{blue}n^{\frac{n^2 + n}{2}}} &
			\end{flalign*}
			\newpage
			\item[C.2] $\text{Let the Group be } G = (P(X), \Delta)$
			\begin{enumerate}
				\setlength{\itemsep}{5pt}
				\item[1.] Checking for closure
				\begin{flalign*}
					&\text{For } A,B \in P(X) \Rightarrow A,B \subseteq X &\\
					&\Rightarrow A\backslash B\subseteq X, B\backslash A \subseteq X &\\
					&\Rightarrow (A\backslash B)\cup(B\backslash A) \subseteq X&\\
					&\Rightarrow (A\backslash B)\cup(B\backslash A) \in P(X)&\\
					&\Rightarrow A\Delta B \in P(X) &
				\end{flalign*}
				$\therefore \Delta$ is closed. 
				\item[2.] Checking for associativity
				\begin{flalign*}
					&\text{For } A,B,C\in P(X) \Rightarrow A,B,C \subseteq X&\\
					&\text{Define mutually disjoint sets } a,b,c,x,y,z,t \subseteq X: &\\
					&\text{\qquad}a = A\backslash (B\cup C),\ b = B\backslash (C\cup A),\ c = C\backslash (A\cup B), &\\
					&\text{\qquad}x = (A\cap B)\backslash C,\ y = (B\cap C)\backslash A,\ z = (C\cap A)\backslash B,\ t = A\cap B\cap C &\\
					&\text{So, }A = a\cup x\cup y\cup t,\ B = b\cup x\cup z\cup t,\ C = c\cup y\cup z\cup t&
				\end{flalign*}
				\begin{flalign*}
					\Rightarrow A\Delta B &= ((a\cup x\cup y\cup t)\backslash(b\cup x\cup z\cup t))\cup((b\cup x\cup z\cup t)\backslash(a\cup x\cup y\cup t))&\\ 
					& = (a\cup y)\cup(b\cup z)&\\
					& = (a\cup b\cup y\cup z) &
				\end{flalign*}
				\begin{flalign*}
					\Rightarrow (A\Delta B)\Delta C &= ((a\cup b\cup y\cup z)\backslash (c\cup y\cup z\cup t))\cup ((c\cup y\cup z\cup t)\backslash(a\cup b\cup y\cup z))&\\
					&= a\cup b\cup c\cup t&
				\end{flalign*}
				\begin{flalign*}
					& \text{Similarly, } B\Delta C = (b\cup c\cup x\cup y)&
				\end{flalign*}
				\begin{flalign*}
					\Rightarrow A\Delta (B\Delta C) &= ((a\cup x\cup y\cup t)\backslash(b\cup c\cup x\cup y))\cup((b\cup c\cup x\cup y)\backslash(a\cup x\cup y\cup t))&\\
					&= a\cup b\cup c\cup t&
				\end{flalign*}
				$\Rightarrow (A\Delta B)\Delta C = A \Delta(B\Delta C)$\\
				$\therefore \Delta$ is associative.
				\item[3.] Finding identity element 1
				\begin{flalign*}
					& A, 1 \in P(X): A\Delta 1 = A&\\
					& \text{Define mutually disjoint sets } a,b,c \subseteq X:&\\
					& \text{\qquad} a = A\backslash 1,\ b = 1\backslash A,\ c = A\cap 1 &\\
					& \text{So, } A = a\cup c,\ 1 = b\cup c &\\
					& \text{Since } A\Delta 1 = A,&\\
					& \text{\quad} ((a\cup c)\backslash(b\cup c))\cup((b\cup c)\backslash(a\cup c)) = a\cup c &\\
					& \text{\quad}\Rightarrow a\cup b = a\cup c &\\
					& \text{\quad} b = c =\phi \text{ is a solution for this identity.} &
				\end{flalign*}
				$\therefore 1 = \phi \text{ exists}$
				\item[4.] Finding inverse of A
				\begin{flalign*}
					&\text{Let } A^{-1} \in G \text{ be the inverse of A } \in G&\\
					& \text{Define mutually disjoint sets } a,b,c \subseteq X:&\\
					& \text{\qquad} a = A\backslash A^{-1},\ b = A^{-1}\backslash A,\ c = A\cap A^{-1} &\\
					& \text{So, } A = a\cup c,\ A^{-1} = b\cup c &\\
					& \text{Using } A \Delta A^{-1} = 1&\\
					& ((a\cup c)\backslash(b\cup c))\cup((b\cup c)\backslash(a\cup c)) = \phi&\\
					& \Rightarrow a\cup b = \phi&\\
					& \Rightarrow a = b = \phi&\\
					& \Rightarrow A = A^{-1} = c&
				\end{flalign*}
				$\therefore A^{-1} = A$ is the inverse of $A, \forall A \in G$.
				\item[5.] Checking for commutativity
				\begin{flalign*}
					&\text{Take } A, B \in G&\\
					& \text{Define mutually disjoint sets } a,b,c \subseteq X:&\\
					& \text{\qquad} a = A\backslash B,\ b = B\backslash A,\ c = A\cap B &\\
					& \text{So, } A = a\cup c,\ B = b\cup c &
				\end{flalign*}
				\begin{flalign*}
					A \Delta B &= ((a\cup c)\backslash(b\cup c))\cup((b\cup c)\backslash(a\cup c))&\\
					&= a\cup b&
				\end{flalign*}
				\begin{flalign*}
					\text{And, } B \Delta A &= ((b\cup c)\backslash(a\cup c))\cup ((a\cup c)\backslash(b\cup c))&\\
					&= b\cup a&\\
					&= a\cup b&
				\end{flalign*}
				$\Rightarrow A\Delta B = B\Delta A$\\
				$\therefore \Delta$ is commutative.
			\end{enumerate}
			{\color{blue}$\therefore G=(P(X), \Delta)$ is a commutative group.}
			\newpage
			\item[C.3] Let $\phi(m) = k, \Zstar{m} = \{a_1, a_2\ldots a_k\}$
			\begin{flalign*}
				& \text{\quad}\Rightarrow \text{By definition, }gcd(a_i, m) = 1 \ \forall \ i \in [k]&\\
				& \forall i \in [k],&\\
				& \forall a \in \Zstar{m},\ gcd(aa_i, m) = 1\text{\qquad} (gcd(a,m)=1 \text{ and } gcd(a_i, m)=1)&\\
				& \text{Since } \forall x \in \Z{m},\ gcd(x, m) = 1\Rightarrow x \in \Zstar{m} (\text{by definition}) \text{ and } gcd(aa_i,m)=gcd(m, aa_i\%m)&\\
				& \Rightarrow \exists j \in [k] \text{ such that } aa_i \equiv a_j \Mod{m}&\\
				& \text{Multiplication by } a \text{ permutes } a_i&\\
				& \Rightarrow (aa_1)(aa_2)\ldots(aa_k) \equiv (a_1)(a_2)\ldots(a_k) \Mod{m} &\\
				& \Rightarrow a^k\cdot(a_1)(a_2)\ldots(a_k) \equiv (a_1)(a_2)\ldots(a_k) \Mod{m} &\\
				& \text{Since } a_i^{-1}\Mod{m} \text{ exists } (gcd(a_i, m) = 1) \text{, multiplying by }(a_1^{-1})(a_2^{-1})\ldots(a_k^{-1}) \Rightarrow a^k \equiv 1 \Mod{m}&
			\end{flalign*}
			{\color{blue}$\therefore a^{\phi(m)} \equiv 1\Mod{m}$}
			\clearpage
		\end{description}
	\end{framed}
	%\end{comment}
	\clearpage
	\item[D. [Symmetric Key Encryption+Classical Ciphers : \!\!\!\!  $\mathbf{30\ points.}$\!\!\!]]
		
	%\begin{comment}
	\begin{framed}
		{\bf Answers}
		\begin{description}
			\item[D.1] There are 4 types of adversarial attacks:
			\begin{itemize}
				\item[1.] Ciphertext-Only Attack (COA)\\
				The attacker knows a collection of ciphertexts through which he/she tries to find the encryption key and plaintexts.
				\item[2.] Known Plaintext Attack (KPA) \\
				The attacker knows a collection of plaintext-ciphertext pairs and uses them to find the encryption key.
				\item[3.] Chosen Plaintext Attack (CPA) \\
				The attacker is able to obtain the corresponding ciphertexts for some chosen plaintexts, to achieve the ultimate goal of finding the encryption key.
				\item[4.] Chosen Ciphertext Attack (CCA) \\
				The attacker already has some plaintext-ciphertext pairs and is further able to obtain the corresponding plaintexts for some chosen ciphertexts. The aim is to find the encryption key.
			\end{itemize}
			{\color{blue}The weakest one among these is COA and the strongest is CCA.}\\
			To help against exploitation by CPA and CCA, randomized encryption algorithms are absolutely needed. This makes prior knowledge of plaintext-ciphertext pairs less useful. If the algorithm is deterministic, the attacker can throw some carefully crafted plaintexts(ciphertexts) and obtain the ciphertexts(plaintexts) making it easy to recover the key.
			\newpage
			\item[D.2] Finding frequency of characters in $c$ and sorting by it yields:\\
			\begin{center}
				\begin{tabular}{|c | c|}
					\hline
					letter & frequency \\
					\hline
					q & 10\\
					\hline
					h & 7\\
					\hline
					c & 6\\
					\hline
					b & 5\\
					\hline
					w & 5\\
					\hline
					r & 5\\
					\hline
					v & 4 \\
					\hline
					n & 3 \\
					\hline
					l & 3 \\
					\hline
					i & 2 \\
					\hline
					d & 2\\
					\hline
					j & 1\\
					\hline
					a & 1\\
					\hline
					s & 1\\
					\hline
					e & 1\\
					\hline
					z & 1\\
					\hline
				\end{tabular}
			\end{center}
			It is clear to see that $q$ can be mapped to $e$ since it is most frequent by a significant difference.\\
			$h$ could be mapped to $t$, the next frequent letter. However, it is seen that $h$ also appears in the middle of a few words, hinting that it is a vowel. So we will map it to $a$, the next frequent vowel.\\
			$c$ has a high frequency and appears only at ends of words and sometimes in doublets, indicating that it is a consonant and it could be $l$\\
			Current state after replacing the mappings with uppercase letters : {\it iE ifLL vEEr fb rdE vfccLE nh rdE LfjwAwz Ar bnnbALL AwwAbsEvEbre AwE vAcE}\\
			On hit and trial, the first two words only make sense if $i$ is replaced by $w$ and then $f$ is mapped to $i$.\\
			Current state after replacing the mappings with uppercase letters : {\it WE WILL vEEr Ib rdE vIccLE nh rdE LIjwAwz Ar bnnbALL AwwAbsEvEbre AwE vAcE}\\
			{\it Ar} should be mapped to {\it AT} to make sense. So, $r$ is $t$.\\
			Doing similar analysis and hit and trials, the plain text comes to be:\\
			{\color{blue}\it we will meet in the middle of the library at noon all arrangements are made}
			\newpage
			\item[D.3] Inverse permutation $\sigma^{-1}(\cdot)$ is:
			\begin{center}
				\begin{tabular}{|c | c | c | c | c | c | c|}
					\hline
					$y$ & 1 & 2 & 3 & 4 & 5 & 6\\
					\hline
					$\sigma^{-1}(y)$ & 4 & 6 & 2 & 3 & 1 & 5\\
					\hline
				\end{tabular}
			\end{center}
			Computing the permutation matrix $K_\sigma$,\\
			$K_\sigma = 
				\begin{bmatrix}
					0 & 0 & 0 & 1 & 0 & 0\\
					0 & 0 & 0 & 0 & 0 & 1\\
					0 & 1 & 0 & 0 & 0 & 0\\
					0 & 0 & 1 & 0 & 0 & 0\\
					1 & 0 & 0 & 0 & 0 & 0\\
					0 & 0 & 0 & 0 & 1 & 0
				\end{bmatrix}
			$\\
			Computing the permutation matrix $K_{\sigma^{-1}}, $\\
			$K_{\sigma^{-1}} =
				\begin{bmatrix}
					0 & 0 & 0 & 0 & 1 & 0\\
					0 & 0 & 1 & 0 & 0 & 0\\
					0 & 0 & 0 & 1 & 0 & 0\\
					1 & 0 & 0 & 0 & 0 & 0\\
					0 & 0 & 0 & 0 & 0 & 1\\
					0 & 1 & 0 & 0 & 0 & 0\\
				\end{bmatrix}
			$\\
			Taking blocks of $t=6$\\
			$m_0 = 
			\begin{bmatrix}
				9 & 20 & 9 & 19 & 20 & 18
			\end{bmatrix}
			\qquad (itistr)
			$\\
			$\Rightarrow c_0 = m_0\cdot K_\sigma =  
			\begin{bmatrix}
				9 & 20 & 9 & 19 & 20 & 18
			\end{bmatrix}
			\cdot
			\begin{bmatrix}
				0 & 0 & 0 & 1 & 0 & 0\\
				0 & 0 & 0 & 0 & 0 & 1\\
				0 & 1 & 0 & 0 & 0 & 0\\
				0 & 0 & 1 & 0 & 0 & 0\\
				1 & 0 & 0 & 0 & 0 & 0\\
				0 & 0 & 0 & 0 & 1 & 0
			\end{bmatrix}
			=
			\begin{bmatrix}
				19\\18\\20\\9\\9\\20
			\end{bmatrix}
			$
			\quad (srtiit)\\
			$m_1 = 
			\begin{bmatrix}
				21 & 5 & 20 & 8 & 1 & 20
			\end{bmatrix}
			\qquad (uethat)
			$\\
			$\Rightarrow c_1 = m_1\cdot K_\sigma =  
			\begin{bmatrix}
				21 & 5 & 20 & 8 & 1 & 20
			\end{bmatrix}
			\cdot
			\begin{bmatrix}
				0 & 0 & 0 & 1 & 0 & 0\\
				0 & 0 & 0 & 0 & 0 & 1\\
				0 & 1 & 0 & 0 & 0 & 0\\
				0 & 0 & 1 & 0 & 0 & 0\\
				1 & 0 & 0 & 0 & 0 & 0\\
				0 & 0 & 0 & 0 & 1 & 0
			\end{bmatrix}
			=
			\begin{bmatrix}
				8\\20\\5\\20\\21\\1
			\end{bmatrix}
			$ 
			\quad (htetua)\\
			$\Rightarrow c_2 = m_2\cdot K_\sigma =  
			\begin{bmatrix}
				20 & 8 & 9 & 19 & 16 & 18
			\end{bmatrix}
			\cdot
			\begin{bmatrix}
				0 & 0 & 0 & 1 & 0 & 0\\
				0 & 0 & 0 & 0 & 0 & 1\\
				0 & 1 & 0 & 0 & 0 & 0\\
				0 & 0 & 1 & 0 & 0 & 0\\
				1 & 0 & 0 & 0 & 0 & 0\\
				0 & 0 & 0 & 0 & 1 & 0
			\end{bmatrix}
			=
			\begin{bmatrix}
				19\\18\\8\\9\\20\\16
			\end{bmatrix}
			$ 
			\quad (srhitp)\\
			$\Rightarrow c_3 =  
			\begin{bmatrix}
				15 & 16 & 15 & 19 & 9 & 20
			\end{bmatrix}
			\cdot
			\begin{bmatrix}
				0 & 0 & 0 & 1 & 0 & 0\\
				0 & 0 & 0 & 0 & 0 & 1\\
				0 & 1 & 0 & 0 & 0 & 0\\
				0 & 0 & 1 & 0 & 0 & 0\\
				1 & 0 & 0 & 0 & 0 & 0\\
				0 & 0 & 0 & 0 & 1 & 0
			\end{bmatrix}
			=
			\begin{bmatrix}
				19\\15\\19\\15\\20\\16
			\end{bmatrix}
			$ 
			\quad (stpooi)\\
			$\Rightarrow c_4 =  
			\begin{bmatrix}
				9 & 15 & 14 & 9 & 19 & 14
			\end{bmatrix}
			\cdot
			\begin{bmatrix}
				0 & 0 & 0 & 1 & 0 & 0\\
				0 & 0 & 0 & 0 & 0 & 1\\
				0 & 1 & 0 & 0 & 0 & 0\\
				0 & 0 & 1 & 0 & 0 & 0\\
				1 & 0 & 0 & 0 & 0 & 0\\
				0 & 0 & 0 & 0 & 1 & 0
			\end{bmatrix}
			=
			\begin{bmatrix}
				19\\14\\9\\9\\15
			\end{bmatrix}
			$ 
			\quad (inonis)\\
			$\Rightarrow c_5 =  
			\begin{bmatrix}
				15 & 20 & 20 & 18 & 21 & 5
			\end{bmatrix}
			\cdot
			\begin{bmatrix}
				0 & 0 & 0 & 1 & 0 & 0\\
				0 & 0 & 0 & 0 & 0 & 1\\
				0 & 1 & 0 & 0 & 0 & 0\\
				0 & 0 & 1 & 0 & 0 & 0\\
				1 & 0 & 0 & 0 & 0 & 0\\
				0 & 0 & 0 & 0 & 1 & 0
			\end{bmatrix}
			=
			\begin{bmatrix}
				21\\20\\18\\15\\5\\20
			\end{bmatrix}
			$ 
			\quad (rettou)\\
			$\therefore$ Ciphertext: {\color{blue}sr ti itht etua srhi tpstpooiino ni sre ttou}\\
			To decrypt, taking $t=6$\\
			$m_0 = c_0\cdot K_\sigma^{-1} =
			\begin{bmatrix}
				19&1&20&9&9&20
			\end{bmatrix}
			\cdot
			\begin{bmatrix}
				0 & 0 & 0 & 0 & 1 & 0\\
				0 & 0 & 1 & 0 & 0 & 0\\
				0 & 0 & 0 & 1 & 0 & 0\\
				1 & 0 & 0 & 0 & 0 & 0\\
				0 & 0 & 0 & 0 & 0 & 1\\
				0 & 1 & 0 & 0 & 0 & 0\\
			\end{bmatrix}
			=
			\begin{bmatrix}
				9\\20\\9\\19\\20\\18
			\end{bmatrix}
			$
			\quad (itistr)\\
			$m_1 = c_1\cdot K_\sigma^{-1} =
			\begin{bmatrix}
				8&20&5&20&21&1
			\end{bmatrix}
			\cdot
			\begin{bmatrix}
				0 & 0 & 0 & 0 & 1 & 0\\
				0 & 0 & 1 & 0 & 0 & 0\\
				0 & 0 & 0 & 1 & 0 & 0\\
				1 & 0 & 0 & 0 & 0 & 0\\
				0 & 0 & 0 & 0 & 0 & 1\\
				0 & 1 & 0 & 0 & 0 & 0\\
			\end{bmatrix}
			=
			\begin{bmatrix}
				21\\5\\20\\8\\1\\20
			\end{bmatrix}
			$
			\quad (uethat)\\
			$m_2 =
			\begin{bmatrix}
				19&18&8&9&20&16
			\end{bmatrix}
			\cdot
			\begin{bmatrix}
				0 & 0 & 0 & 0 & 1 & 0\\
				0 & 0 & 1 & 0 & 0 & 0\\
				0 & 0 & 0 & 1 & 0 & 0\\
				1 & 0 & 0 & 0 & 0 & 0\\
				0 & 0 & 0 & 0 & 0 & 1\\
				0 & 1 & 0 & 0 & 0 & 0\\
			\end{bmatrix}
			=
			\begin{bmatrix}
				20\\8\\9\\19\\16\\18
			\end{bmatrix}
			$
			\quad (thispr)\\
			$m_3 =
			\begin{bmatrix}
				15\\16\\15\\19\\9\\20
			\end{bmatrix}
			$
			\quad (oposit)\\
			$m_4 =
			\begin{bmatrix}
				9\\15\\14\\9\\19\\14
			\end{bmatrix}
			$
			\quad (ionisn)\\
			$m_5 =
			\begin{bmatrix}
				15\\20\\20\\18\\21\\5
			\end{bmatrix}
			$
			\quad (ottrue)\\
			$\therefore m =$ {\it it is true that this proposition is not true}\\
			$\Rightarrow$ Decryption verified.
			\clearpage
			\item[D.4] Computing order of $GL_t(\Z{p})$\\
			Let $A =
			\begin{bmatrix}
				v_1 \\ v_2 \\ \vdots \\ v_t
			\end{bmatrix}
			\in GL_t(\Z{p}),\ v_i \in \Z{p}^n\ \forall\ i \in [t]$\\
			Number of ways to choose $v_1 = p^t - 1$ \quad (Excluding zero vector)\\
			Number of ways to choose $v_2 = p^t - p$ \quad (Excluding multiples of $v_1$)\\
			\vdots\\
			Number of ways to choose $v_n = p^t - p^{t-1}$\\
			Number of ways to choose $A = (p^t - 1)(p^t - p)\ldots(p^t-p^{t-1})$\\
			{\color{blue}$\therefore$ order of $GL_t(\Z{p}) = (p^t - 1)(p^t - p)\ldots(p^t-p^{t-1})$}\\
			Order of $GL_2(\Z{26}) = (26^2 - 1)\cdot(26^2-26) = 438750$
			\begin{flalign*}
				\text{Number of valid keys with even determinants} &= \text{Number of valid keys with at least 1 even row}&\\
				&= 2\cdot(13^2 - 1)\cdot(26^2-13^2) + (13^2-1)\cdot(13^2-26)&\\
				&= 2\cdot168\cdot169 + 168\cdot143&\\
				&= 194376&
			\end{flalign*}
			{\color{blue} There are $194376$ valid keys with even determinants.}\\ \\
			Computing the number for $GL_2(\Z{p})$
			\begin{flalign*}
				\text{Number of valid keys with even determinants} &= \text{Number of valid keys with at least 1 even row}&\\
				&= 2\cdot(\left\lceil\frac{p}{2}\right\rceil^2 - 1)\cdot(p^2-\left\lfloor\frac{p}{2}\right\rfloor^2) + (\left\lceil\frac{p}{2}\right\rceil^2-1)\cdot(\left\lceil\frac{p}{2}\right\rceil^2-p)&\\
				&= 2\cdot(\left\lceil\frac{p}{2}\right\rceil^2 - 1)\cdot(p^2-(p - \left\lceil\frac{p}{2}\right\rceil)^2) + (\left\lceil\frac{p}{2}\right\rceil^2-1)\cdot(\left\lceil\frac{p}{2}\right\rceil^2-p)&\\
				&= 4\cdot p\cdot\left\lceil\frac{p}{2}\right\rceil^3 - \left\lceil\frac{p}{2}\right\rceil^4 - 4\cdot p \cdot\left\lceil\frac{p}{2}\right\rceil - (p-1)\cdot \left\lceil\frac{p}{2}\right\rceil^2 + p&\\
			\end{flalign*}
		\end{description}
	\end{framed}
	%\end{comment}
\end{description}
\end{document}
